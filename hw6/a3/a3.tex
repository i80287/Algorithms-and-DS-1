\documentclass{report}

%%%%%%%%%%%%%%%%%%%%%%%%%%%%%%%%%
% PACKAGE IMPORTS
%%%%%%%%%%%%%%%%%%%%%%%%%%%%%%%%%

\usepackage[utf8]{inputenc} % Сурс - семинары по алгебре Медведя Никиты Юрьевича
\usepackage[T2A]{fontenc}

\usepackage[tmargin=2cm,rmargin=1in,lmargin=1in,margin=0.85in,bmargin=2cm,footskip=.2in]{geometry}
\usepackage{amsmath,amsfonts,amsthm,amssymb,mathtools}
\usepackage[varbb]{newpxmath}
\usepackage{xfrac}
\usepackage[makeroom]{cancel}
\usepackage{mathtools}
\usepackage{bookmark}
\usepackage{enumitem}
\usepackage{hyperref,theoremref}
\hypersetup{
	pdftitle={Assignment},
	colorlinks=true, linkcolor=doc!90,
	bookmarksnumbered=true,
	bookmarksopen=true
}
\usepackage[most,many,breakable]{tcolorbox}
\usepackage{xcolor}% http://ctan.org/pkg/xcolor
%\usepackage{colortbl}% http://ctan.org/pkg/colortbl
\usepackage{multirow}% http://ctan.org/pkg/multirow
\usepackage{graphicx}% http://ctan.org/pkg/graphicx
\usepackage{varwidth}
\usepackage{varwidth}
\usepackage{etoolbox}
%\usepackage{authblk}
\usepackage{nameref}
\usepackage{multicol,array}
\usepackage{tikz-cd}
\usepackage[ruled,vlined,linesnumbered]{algorithm2e}
\usepackage{comment} % enables the use of multi-line comments (\ifx \fi) 
\usepackage{import}
\usepackage{xifthen}
\usepackage{pdfpages}
\usepackage{transparent}

\usepackage[english]{babel}
% ,russian
\usepackage{amsfonts,amssymb}
\usepackage{relsize}

\usepackage{systeme}

\usepackage{indentfirst} % Красная строка
\usepackage{fancyhdr}
\usepackage{wrapfig}
\usepackage{textcomp}

% \usepackage[unicode]{hyperref}

\newcommand\mycommfont[1]{\footnotesize\ttfamily\textcolor{blue}{#1}}
\SetCommentSty{mycommfont}
\newcommand{\incfig}[1]{%
    \def\svgwidth{\columnwidth}
    \import{./figures/}{#1.pdf_tex}
}

\usepackage{tikzsymbols}
\renewcommand\qedsymbol{$\Laughey$}


%\usepackage{import}
%\usepackage{xifthen}
%\usepackage{pdfpages}
%\usepackage{transparent}


%%%%%%%%%%%%%%%%%%%%%%%%%%%%%%
% SELF MADE COLORS
%%%%%%%%%%%%%%%%%%%%%%%%%%%%%%



\definecolor{myg}{RGB}{56, 140, 70}
\definecolor{myb}{RGB}{45, 111, 177}
\definecolor{myr}{RGB}{199, 68, 64}
\definecolor{mytheorembg}{HTML}{F2F2F9}
\definecolor{mytheoremfr}{HTML}{00007B}
\definecolor{mylenmabg}{HTML}{FFFAF8}
\definecolor{mylenmafr}{HTML}{983b0f}
\definecolor{mypropbg}{HTML}{f2fbfc}
\definecolor{mypropfr}{HTML}{191971}
\definecolor{myexamplebg}{HTML}{F2FBF8}
\definecolor{myexamplefr}{HTML}{88D6D1}
\definecolor{myexampleti}{HTML}{2A7F7F}
\definecolor{mydefinitbg}{HTML}{E5E5FF}
\definecolor{mydefinitfr}{HTML}{3F3FA3}
\definecolor{notesgreen}{RGB}{0,162,0}
\definecolor{myp}{RGB}{197, 92, 212}
\definecolor{mygr}{HTML}{2C3338}
\definecolor{myred}{RGB}{127,0,0}
\definecolor{myyellow}{RGB}{169,121,69}
\definecolor{myexercisebg}{HTML}{F2FBF8}
\definecolor{myexercisefg}{HTML}{88D6D1}

%%%%%%%%%%%%%%%%%%%%%%%%%%%%%%%%%%%%%%%%%%%
% TABLE OF CONTENTS
%%%%%%%%%%%%%%%%%%%%%%%%%%%%%%%%%%%%%%%%%%%

\usepackage{tikz}
\definecolor{doc}{RGB}{0,60,110}
\usepackage{titletoc}
\contentsmargin{0cm}
\titlecontents{chapter}[3.7pc]
{\addvspace{30pt}%
	\begin{tikzpicture}[remember picture, overlay]%
		\draw[fill=doc!60,draw=doc!60] (-7,-.1) rectangle (-0.9,.5);%
		\pgftext[left,x=-3.5cm,y=0.2cm]{\color{white}\Large\sc\bfseries Chapter\ \thecontentslabel};%
	\end{tikzpicture}\color{doc!60}\large\sc\bfseries}%
{}
{}
{\;\titlerule\;\large\sc\bfseries Page \thecontentspage
	\begin{tikzpicture}[remember picture, overlay]
		\draw[fill=doc!60,draw=doc!60] (2pt,0) rectangle (4,0.1pt);
	\end{tikzpicture}}%
\titlecontents{section}[3.7pc]
{\addvspace{2pt}}
{\contentslabel[\thecontentslabel]{2pc}}
{}
{\hfill\small \thecontentspage}
[]
\titlecontents*{subsection}[3.7pc]
{\addvspace{-1pt}\small}
{}
{}
{\ --- \small\thecontentspage}
[ \textbullet\ ][]

\makeatletter
\renewcommand{\tableofcontents}{%
	\chapter*{%
	  \vspace*{-20\p@}%
	  \begin{tikzpicture}[remember picture, overlay]%
		  \pgftext[right,x=15cm,y=0.2cm]{\color{doc!60}\Huge\sc\bfseries \contentsname};%
		  \draw[fill=doc!60,draw=doc!60] (13,-.75) rectangle (20,1);%
		  \clip (13,-.75) rectangle (20,1);
		  \pgftext[right,x=15cm,y=0.2cm]{\color{white}\Huge\sc\bfseries \contentsname};%
	  \end{tikzpicture}}%
	\@starttoc{toc}}
\makeatother


\usepackage{ragged2e}
\usepackage{blindtext}
\usepackage{booktabs}
\input{macros}

\begin{document}

\chapter*{Пункт 1.}

\raggedright

Рассмотрим алгоритм, который ищет множество вершин, убрав которые из графа $G$ вершина $t$ перестанет быть достижимой из вершины $s$:

Докажем простую лемму (быть может, можно было просто сказать "очевидно, что")

\mclemma{Лемма 1}{
    Пусть дан ориентированный граф $G = (V, E), t \in V, s \in V, s \ne t$
    
    Тогда вершина $t$ не достижима из вершины $s \iff $ в графе $G$ нет простых путей из $s$ в $t$
\mcprf{
\begin{split}
    & "\implies" \\
    & \text{Т.к. вершина $t$ не достижима из вершины $s$, то в $G$ нет путей из $s$ в $t \implies $ нет простых путей из $s$ в $t$} \\
    & "\impliedby" \\
    & \text{Предположим от противного, т.е. в графе $G$ нет простых путей из $s$ в $t$, но вершина $t$} \\
    & \text{достижима из вершины $s$} \\
    & \text{Тогда существует путь $P$ из $s$ в $t$, который не является простым. Это значит, что в нём} \\
    & \text{какая-то вершина встречается более 1 раза, т.е. путь имеет вид } P = (s, v_1, ..., v_i, q_1, q_2, ... q_k, v_i, ..., t) \\
    & \text{Удалим цикл $ (v_i, q_1, q_2, ... q_k, v_i) $ из пути $P$, получив новый путь $P'$ из $s$ в $t$} \\
    & \text{Если в нём всё ещё есть циклы, будем аналогично удалять их} \\
    & \text{(их число конечно, т.к. длина пути конечна)} \\
    & \text{В итоге получим путь из $s$ в $t$, в котором нет циклов, т.е. простой путь,} \\
    & \text{ но по предположению их нет $\implies \bot$} \\
    & \text{Следовательно, предположение не верно и $t$ не достижима из $s$} \\
\end{split}
}
}

Назовём множество вершин $M \subseteq V$ "хорошим", если это минимальное по мощности множество вершин, которые достаточно удалить, чтобы $t$ перестала быть достижимой из $s$

Докажем простую лемму (быть может, можно было просто сказать "очевидно, что")

\mclemma{Лемма 2}{
    Пусть $M$ - "хорошее" множество

    1. Любой простой путь из $s$ в $t$ содержит хотя бы 1 вершину из $M$
    2. Любая вершина из $M$ содержится хотя бы в 1 простом пути из $s$ в $t$
\mcprf{
\begin{split}
    & \text{1. Предположим от противного, т.е. $\exists$ простой путь $P$ из $s$ в $t$, который не проходит} \\
    & \text{ни через одну вершину из $M$} \\
    & \text{Удалим из графа $G$ все вершины множества $M$. По определению множества $M$ после этого} \\
    & \text{вершина $t$ должна перестать быть достижимой из $s$} \\
    & \text{Но т.к. ни одна вершина из $M$ не принадлежит пути $P$, то этот путь останется в графе и} \\
    & \text{вершина $t$ всё ещё будет достижима из $s \implies$ противоречие с определением множества $M \implies$} \\
    & \implies \text{предположение о существовании такого пути неверно} \\
    & \text{1. Предположим от противного, т.е. $\exists$ вершина $v \in M$, т.ч. она не принадлежит} \\
    & \text{ни одному простому пути.} \\
    & \text{Тогда эта вершина не принадлежит ни одному пути из $s$ в $t$. Но тогда её наличие / отсутствие} \\
    & \text{в графе $G$ не влияет на достижимость $t$ из $s$. Удалим эту вершину из $M$, получив $M'$} \\
    & \text{Множество $M'$ удовлетворяет определению "хорошего" множества, но $|M'| = |M| - 1 \implies$ } \\
    & \text{$\implies$ множество $M$ не является "хорошим" $\implies \bot \implies $ предположение о существовании} \\
    & \text{такой вершины $v$ неверно} \\
\end{split}
}
}

Из лемм 1 и 2 следует, что необходимым и достаточным условием "разрыва" достижимости $t$ из $s$ будет:
каким-либо образом удалить все простые пути из $s$ в $t$

Тогда для данного графа $G$ выпишем все его простые пути из $s$ в $t$:

\begin{align*}
    & (s, v_1, v_2, \dots, v_k, t) \\
    & (s, w_1, w_2, \dots, w_n, t) \\
    & (s, q_1, q_2, \dots, q_m, t) \\
    & (s, p_1, p_2, \dots, p_l, t) \\
    & \dots \\
\end{align*}

Посчитаем для каждой вершины, в каком количестве путей она встречается, обозначим этот массив $count[]$ (вершина - число от $0$ до $|V| - 1$)

И на каждой итерации алгоритма:

Удалим пути, которые содержат вершину с самой высокой частотой встречаемости (т.е. с наибольшим $count[v_i]$)

При этом, при удалении каждого пути, пройдёмся по нему и обновим (уменьшим значения в $count[]$) для каждой вершины из пути

Удалённную вершину добавим в множество $M$

Когда не останется простых путей, вернём $|M|$

Асимптотика работы алгоритма - $ \mathcal{O}(|V| \cdot 2^{|V|}) $, т.к. для поиска простых путей будем использовать $bfs$, поддерживая для каждого пути, какие вершины в нём уже есть
(массива $visited[]$ нет, $bfs$ не добавляет вершину, только если она есть во всех путях или длина текущего пути $\ge |V|$)

\par

%  графе $G = (V, E)$

% $\begin{tikzpicture}
%     \node[draw,align=left] at (0, 1) {Граф G:};
%     \coordinate (v1) at (0, -1);
%     \coordinate (v2) at (1, 0);
%     \coordinate (v3) at (1, -2);
%     \coordinate (v4) at (3, -2);
%     \coordinate (v5) at (2, 0);
%     \coordinate (v6) at (2.75, -0.5);
%     \coordinate (v7) at (3.75, -1);
%     \coordinate (v8) at (5.5, -0.25);
%     \coordinate (v9) at (2, -3);
%     \filldraw[black] (v1) circle (1pt) node[anchor=east]{$v_1$};
%     \filldraw[black] (v2) circle (1pt) node[anchor=south]{$v_2$};
%     \filldraw[black] (v3) circle (1pt) node[anchor=north]{$v_3$};
%     \filldraw[black] (v4) circle (1pt) node[anchor=north west]{$v_4$};
%     \filldraw[black] (v5) circle (1pt) node[anchor=south]{$v_5$};
%     \filldraw[black] (v6) circle (1pt) node[anchor=south]{$v_6$};
%     \filldraw[black] (v7) circle (1pt) node[anchor=south]{$v_7$};
%     \filldraw[black] (v8) circle (1pt) node[anchor=west]{$v_8$};
%     \filldraw[black] (v9) circle (1pt) node[anchor=north]{$v_9$};
%     \draw[-stealth,line width=0.4mm] (v1) -- node[anchor=south]{2} (v2);
%     \draw[-stealth,line width=0.4mm] (v1) -- node[anchor=west]{3} (v3);
%     \draw[-stealth,line width=0.4mm] (v3) -- node[anchor=north]{7} (v4);
%     \draw[-stealth,line width=0.4mm] (v2) -- node[anchor=south]{3} (v5);
%     \draw[-stealth,line width=0.4mm] (v2) -- node[anchor=north east]{2} (v4);
%     \draw[-stealth,line width=0.4mm] (v5) -- node[anchor=north east]{2} (v6);
%     \draw[-stealth,line width=0.4mm] (v6) -- node[anchor=west]{5} (v4);
%     \draw[-stealth,line width=0.4mm] (v6) -- node[anchor=south]{2} (v7);
%     \draw[-stealth,line width=0.4mm] (v4) -- node[anchor=north]{7} (v8);
%     \draw[-stealth,line width=0.4mm] (v5) -- node[anchor=south]{5} (v8);
%     \draw[-stealth,line width=0.4mm] (v7) -- node[anchor=south east]{1} (v8);
%     \draw[-stealth,line width=0.4mm] (v3) -- node[anchor=north east]{3} (v9);
%     \draw[-stealth,line width=0.4mm] (v9) -- node[anchor=north west]{2} (v4);
% \end{tikzpicture} $

% \begin{align*}
%     & \text{Изначально } dist = [ 1, +\infty, +\infty, +\infty, +\infty, +\infty, +\infty, +\infty, +\infty ] \\
%     & \text{После посещения } v_1: dist = [ 1, 2, 3, +\infty, +\infty, +\infty, +\infty, +\infty, +\infty, ] \\
%     & \text{После посещения } v_2: dist = [ 1, 2, 3, 4, 6, +\infty, +\infty, +\infty, +\infty, ] \\
%     & \text{После посещения } v_3: dist = [ 1, 2, 3, 4, 6, +\infty, +\infty, +\infty, 9, ] \\
%     & \text{После посещения } v_4: dist = [ 1, 2, 3, 4, 6, +\infty, +\infty, 28, 9, ] \\
%     & \text{После посещения } v_5: dist = [ 1, 2, 3, 4, 6, 12, +\infty, 28, 9, ] \\
%     & \text{После посещения } v_9: dist = [ 1, 2, 3, 4, 6, 12, +\infty, 28, 9, ] \\
%     & \text{После посещения } v_6: dist = [ 1, 2, 3, 4, 6, 12, 24, 28, 9, ] \\
%     & \text{После посещения } v_7: dist = [ 1, 2, 3, 4, 6, 12, 24, 24, 9, ] \\
%     & \text{После посещения } v_8: dist = [ 1, 2, 3, 4, 6, 12, 24, 24, 9, ] \\
%     & \text{Минимальный путь из $v_1$ в $v_8$ - путь } (v_1, v_2, v_5, v_6, v_7, v_8) \text{, вес которого равен 24} \\
% \end{align*}

\end{document}
