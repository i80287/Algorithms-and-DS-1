\documentclass{report}

%%%%%%%%%%%%%%%%%%%%%%%%%%%%%%%%%
% PACKAGE IMPORTS
%%%%%%%%%%%%%%%%%%%%%%%%%%%%%%%%%

\usepackage[utf8]{inputenc} % Сурс - семинары по алгебре Медведя Никиты Юрьевича
\usepackage[T2A]{fontenc}

\usepackage[tmargin=2cm,rmargin=1in,lmargin=1in,margin=0.85in,bmargin=2cm,footskip=.2in]{geometry}
\usepackage{amsmath,amsfonts,amsthm,amssymb,mathtools}
\usepackage[varbb]{newpxmath}
\usepackage{xfrac}
\usepackage[makeroom]{cancel}
\usepackage{mathtools}
\usepackage{bookmark}
\usepackage{enumitem}
\usepackage{hyperref,theoremref}
\hypersetup{
	pdftitle={Assignment},
	colorlinks=true, linkcolor=doc!90,
	bookmarksnumbered=true,
	bookmarksopen=true
}
\usepackage[most,many,breakable]{tcolorbox}
\usepackage{xcolor}% http://ctan.org/pkg/xcolor
%\usepackage{colortbl}% http://ctan.org/pkg/colortbl
\usepackage{multirow}% http://ctan.org/pkg/multirow
\usepackage{graphicx}% http://ctan.org/pkg/graphicx
\usepackage{varwidth}
\usepackage{varwidth}
\usepackage{etoolbox}
%\usepackage{authblk}
\usepackage{nameref}
\usepackage{multicol,array}
\usepackage{tikz-cd}
\usepackage[ruled,vlined,linesnumbered]{algorithm2e}
\usepackage{comment} % enables the use of multi-line comments (\ifx \fi) 
\usepackage{import}
\usepackage{xifthen}
\usepackage{pdfpages}
\usepackage{transparent}

\usepackage[english]{babel}
% ,russian
\usepackage{amsfonts,amssymb}
\usepackage{relsize}

\usepackage{systeme}

\usepackage{indentfirst} % Красная строка
\usepackage{fancyhdr}
\usepackage{wrapfig}
\usepackage{textcomp}

% \usepackage[unicode]{hyperref}

\newcommand\mycommfont[1]{\footnotesize\ttfamily\textcolor{blue}{#1}}
\SetCommentSty{mycommfont}
\newcommand{\incfig}[1]{%
    \def\svgwidth{\columnwidth}
    \import{./figures/}{#1.pdf_tex}
}

\usepackage{tikzsymbols}
\renewcommand\qedsymbol{$\Laughey$}


%\usepackage{import}
%\usepackage{xifthen}
%\usepackage{pdfpages}
%\usepackage{transparent}


%%%%%%%%%%%%%%%%%%%%%%%%%%%%%%
% SELF MADE COLORS
%%%%%%%%%%%%%%%%%%%%%%%%%%%%%%



\definecolor{myg}{RGB}{56, 140, 70}
\definecolor{myb}{RGB}{45, 111, 177}
\definecolor{myr}{RGB}{199, 68, 64}
\definecolor{mytheorembg}{HTML}{F2F2F9}
\definecolor{mytheoremfr}{HTML}{00007B}
\definecolor{mylenmabg}{HTML}{FFFAF8}
\definecolor{mylenmafr}{HTML}{983b0f}
\definecolor{mypropbg}{HTML}{f2fbfc}
\definecolor{mypropfr}{HTML}{191971}
\definecolor{myexamplebg}{HTML}{F2FBF8}
\definecolor{myexamplefr}{HTML}{88D6D1}
\definecolor{myexampleti}{HTML}{2A7F7F}
\definecolor{mydefinitbg}{HTML}{E5E5FF}
\definecolor{mydefinitfr}{HTML}{3F3FA3}
\definecolor{notesgreen}{RGB}{0,162,0}
\definecolor{myp}{RGB}{197, 92, 212}
\definecolor{mygr}{HTML}{2C3338}
\definecolor{myred}{RGB}{127,0,0}
\definecolor{myyellow}{RGB}{169,121,69}
\definecolor{myexercisebg}{HTML}{F2FBF8}
\definecolor{myexercisefg}{HTML}{88D6D1}

%%%%%%%%%%%%%%%%%%%%%%%%%%%%%%%%%%%%%%%%%%%
% TABLE OF CONTENTS
%%%%%%%%%%%%%%%%%%%%%%%%%%%%%%%%%%%%%%%%%%%

\usepackage{tikz}
\definecolor{doc}{RGB}{0,60,110}
\usepackage{titletoc}
\contentsmargin{0cm}
\titlecontents{chapter}[3.7pc]
{\addvspace{30pt}%
	\begin{tikzpicture}[remember picture, overlay]%
		\draw[fill=doc!60,draw=doc!60] (-7,-.1) rectangle (-0.9,.5);%
		\pgftext[left,x=-3.5cm,y=0.2cm]{\color{white}\Large\sc\bfseries Chapter\ \thecontentslabel};%
	\end{tikzpicture}\color{doc!60}\large\sc\bfseries}%
{}
{}
{\;\titlerule\;\large\sc\bfseries Page \thecontentspage
	\begin{tikzpicture}[remember picture, overlay]
		\draw[fill=doc!60,draw=doc!60] (2pt,0) rectangle (4,0.1pt);
	\end{tikzpicture}}%
\titlecontents{section}[3.7pc]
{\addvspace{2pt}}
{\contentslabel[\thecontentslabel]{2pc}}
{}
{\hfill\small \thecontentspage}
[]
\titlecontents*{subsection}[3.7pc]
{\addvspace{-1pt}\small}
{}
{}
{\ --- \small\thecontentspage}
[ \textbullet\ ][]

\makeatletter
\renewcommand{\tableofcontents}{%
	\chapter*{%
	  \vspace*{-20\p@}%
	  \begin{tikzpicture}[remember picture, overlay]%
		  \pgftext[right,x=15cm,y=0.2cm]{\color{doc!60}\Huge\sc\bfseries \contentsname};%
		  \draw[fill=doc!60,draw=doc!60] (13,-.75) rectangle (20,1);%
		  \clip (13,-.75) rectangle (20,1);
		  \pgftext[right,x=15cm,y=0.2cm]{\color{white}\Huge\sc\bfseries \contentsname};%
	  \end{tikzpicture}}%
	\@starttoc{toc}}
\makeatother


\usepackage{ragged2e}
\usepackage{blindtext}
\usepackage{booktabs}
\input{macros}

\begin{document}

\chapter*{Задание A2}

\section*{Пункт 1.}

\raggedright
В классическом алгоритме Дейкстры на каждом шаге у вершины $v$, до которой сейчас найден ближайший путь, просматриваем
всех соседей $u$ и обновляем расстояния до них как $dist[u] = \min(dist[u], dist[v] + w(v, u))$.

Пользуясь тем, что уже существует доказательство корректности алгоритма Дейкстры, заметим, что в нём используется то,
что веса рёбер неотрицательные и прибавление $w(v, u)$ к $dist[v]$ может уменьшить $dist[u]$, но не сделает его меньше $dist[v]$.

Таким образом, при модификации алгоритма достаточно заменить операцию сложения на умножение и при инициализации расстояние
до стартовой вершины сделать равным 1, а не 0, тогда алгоритм будет работать корректно, если веса всех рёбер неотрицательны.

Если в графе есть ребро $(u, v)$ с отрицательным весом, то до вершины $u$ можно дойти по очень длинному пути, а потом пройти по ребру $(u, v)$,
дойдя до вершины $v$ по пути с большим отрицательным весом
\par

Рассмотрим пример: пусть в ориентированном графе $G = (V, E)$ необходимо найти кратчайший путь из $v_1$ в $v_8$

$\begin{tikzpicture}
    \node[draw,align=left] at (0, 1) {Граф G:};
    \coordinate (v1) at (0, -1);
    \coordinate (v2) at (1, 0);
    \coordinate (v3) at (1, -2);
    \coordinate (v4) at (3, -2);
    \coordinate (v5) at (2, 0);
    \coordinate (v6) at (2.75, -0.5);
    \coordinate (v7) at (3.75, -1);
    \coordinate (v8) at (5.5, -0.25);
    \coordinate (v9) at (2, -3);
    \filldraw[black] (v1) circle (1pt) node[anchor=east]{$v_1$};
    \filldraw[black] (v2) circle (1pt) node[anchor=south]{$v_2$};
    \filldraw[black] (v3) circle (1pt) node[anchor=north]{$v_3$};
    \filldraw[black] (v4) circle (1pt) node[anchor=north west]{$v_4$};
    \filldraw[black] (v5) circle (1pt) node[anchor=south]{$v_5$};
    \filldraw[black] (v6) circle (1pt) node[anchor=south]{$v_6$};
    \filldraw[black] (v7) circle (1pt) node[anchor=south]{$v_7$};
    \filldraw[black] (v8) circle (1pt) node[anchor=west]{$v_8$};
    \filldraw[black] (v9) circle (1pt) node[anchor=north]{$v_9$};
    \draw[-stealth,line width=0.4mm] (v1) -- node[anchor=south]{2} (v2);
    \draw[-stealth,line width=0.4mm] (v1) -- node[anchor=west]{3} (v3);
    \draw[-stealth,line width=0.4mm] (v3) -- node[anchor=north]{7} (v4);
    \draw[-stealth,line width=0.4mm] (v2) -- node[anchor=south]{3} (v5);
    \draw[-stealth,line width=0.4mm] (v2) -- node[anchor=north east]{2} (v4);
    \draw[-stealth,line width=0.4mm] (v5) -- node[anchor=north east]{2} (v6);
    \draw[-stealth,line width=0.4mm] (v6) -- node[anchor=west]{5} (v4);
    \draw[-stealth,line width=0.4mm] (v6) -- node[anchor=south]{2} (v7);
    \draw[-stealth,line width=0.4mm] (v4) -- node[anchor=north]{7} (v8);
    \draw[-stealth,line width=0.4mm] (v5) -- node[anchor=south]{5} (v8);
    \draw[-stealth,line width=0.4mm] (v7) -- node[anchor=south east]{1} (v8);
    \draw[-stealth,line width=0.4mm] (v3) -- node[anchor=north east]{3} (v9);
    \draw[-stealth,line width=0.4mm] (v9) -- node[anchor=north west]{2} (v4);
\end{tikzpicture} $

\begin{align*}
    & \text{Изначально } dist = [ 1, +\infty, +\infty, +\infty, +\infty, +\infty, +\infty, +\infty, +\infty ] \\
    & \text{После посещения } v_1: dist = [ 1, 2, 3, +\infty, +\infty, +\infty, +\infty, +\infty, +\infty, ] \\
    & \text{После посещения } v_2: dist = [ 1, 2, 3, 4, 6, +\infty, +\infty, +\infty, +\infty, ] \\
    & \text{После посещения } v_3: dist = [ 1, 2, 3, 4, 6, +\infty, +\infty, +\infty, 9, ] \\
    & \text{После посещения } v_4: dist = [ 1, 2, 3, 4, 6, +\infty, +\infty, 28, 9, ] \\
    & \text{После посещения } v_5: dist = [ 1, 2, 3, 4, 6, 12, +\infty, 28, 9, ] \\
    & \text{После посещения } v_9: dist = [ 1, 2, 3, 4, 6, 12, +\infty, 28, 9, ] \\
    & \text{После посещения } v_6: dist = [ 1, 2, 3, 4, 6, 12, 24, 28, 9, ] \\
    & \text{После посещения } v_7: dist = [ 1, 2, 3, 4, 6, 12, 24, 24, 9, ] \\
    & \text{После посещения } v_8: dist = [ 1, 2, 3, 4, 6, 12, 24, 24, 9, ] \\
    & \text{Минимальный путь из $v_1$ в $v_8$ - путь } (v_1, v_2, v_5, v_6, v_7, v_8) \text{, вес которого равен 24} \\
\end{align*}

\section*{Пункт 2.}

\raggedright
Пусть, если вершина $v_j$ не достижима из вершины $v_i$ $\implies$ между ними нет кратчайшего пути, то положим $dist[v_i][v_j] = \infty$

1. Сначала для всех пар $1 \le i, j, k \le n$, для которых $ i \ne j \ne k \ne i $ необходимо проверить выполнение неравенства треугольника:

\[
    \left( dist[v_i][v_j] \ne \infty \wedge dist[v_j][v_k] \ne \infty \right)
    \implies
    \left( dist[v_i][v_k] \ne \infty \wedge dist[v_i][v_j] + dist[v_j][v_k] \ge dist[v_i][v_k] \right)
(1) \]

Если оно не выполняется, то приходим к противоречию с определением кратчайшего пути $ \implies $ по такой матрице $dist[][]$
нельзя восстановить граф.

2. При выполнении неравенства рассмотрим такой алгоритм:

В качестве матрицы весов $w$ восстанавливаемого графа положим данную матрицу $dist[][]$ кратчайших путей

А в качестве восстанавливаемого графа $G$ рассмотрим список смежности, построенный из этой матрицы
(вершина $u$ принадлежит списку вершины $v$, если $dist[u][v] \ne inf$)

Это корректно, т.к. если дан ориентированный взвешенный граф $H$, и к нему применили алгоритм Флойда-Уоршелла, получив
матрицу кратчайших путей $dist_H[][]$, то, применив алгоритм Флойда-Уоршелла уже к этой матрице $dist_H[][]$, проинтерпретировав
её как матрицу весов, алгоритм вернёт ту же матрицу $dist_H$ (по факту, ни на одной итерации алгоритма обновление элементов
матрицы не произойдёт, т.к. между вершинами уже найдены оптимальные пути в виду выполнения неравенства треугольника)

В качестве примера рассмотрим матрицу $dist[][]$:

$$
dist = \begin{bmatrix}
     0 & \infty & 4 & \infty & 2 \\
    -1 &  0 & 3 & \infty & 1 \\
     4 & \infty & 0 & \infty & 6 \\
     2 & \infty & 3 &  0 & 4 \\ 
     1 & \infty & 2 & \infty & 0 \\
\end{bmatrix}
$$

Здесь присутствует путь отрицательного веса между вершинами $v_2$ и $v_1$, а также не между любой парой вершин есть путь

Для всех троек элементов данной матриц выполняется неравенство треугольника в форме (1)

Применив к $dist[][]$ алгоритм Флойда-Уоршелла, получим ту же матрицу $dist[][]$

Тогда для восстанавливаемого графа положим матрицу весов равной $ w = dist $ и сформируем список смежности:

\begin{align*}
    L_G = \{ & \\
        v_1: & \, \{ v_3, v_5 \}, \\
        v_2: & \, \{ v_1, v_3, v_5 \}, \\
        v_3: & \, \{ v_1, v_5 \}, \\
        v_4: & \, \{ v_1, v_3, v_5 \} \\
        v_5: & \, \{ v_1, v_3 \} \\
    \} \hspace*{1cm} & \\
\end{align*}

% Для всех пар $ 1 \le i \ne j \le n $, таких что $dist[v_i][v_j] \ne \infty $:

% Если \[ \forall k: k = i \vee k = j \vee dist[v_i][v_k] = \infty \vee dist[v_k][v_j] = \infty \vee dist[v_i][v_k] + dist[v_k][v_j] > dist[v_i][v_j] \]

% то из вершины $v_i$ нельзя попасть в вершину $v_j$ через какую-либо другую вершину.
% Но при этом $dist[v_i][v_j] \ne \infty \implies (v_i, v_j) \in E $,
% т.е. просто соединяем вершины ориентированным ребром

% Иначе, если \[ \exists k: k \ne i \wedge k \ne j \wedge dist[v_i][v_k] \ne \infty \wedge dist[v_k][v_j] \ne \infty \wedge dist[v_i][v_k] + dist[v_k][v_j] = dist[v_i][v_j] \]

% то через такие вершины $v_k$ можно пройти на пути из $v_i$ в $v_j$. Добавим все такие вершины $v_k$ в список вершин, которые могут быть на пути из $v_i$ в $v_j$

\par

\pagebreak

\section*{Пункт 4.}

\raggedright

Т.к. в графе $G = (V, E)$ есть путь  из $a$ в $b$, проходящий через ребро $(v_i, v_j)$, то из вершины
$a$ есть путь $P_{a, v_i}$ до вершины $v_i$ и из вершины $v_j$ есть путь $P_{v_j, b}$ до вершины $b$.
Аналогично, т.к. в графе $G$ есть путь из $b$ в $a$, проходящий через ребро $(v_i, v_j)$, то из вершины 
$b$ есть путь $P_{b, v_i}$ до вершины $v_i$ и из вершины $v_j$ есть путь $P_{v_j, a}$ до вершины $a$.

Следовательно, в графе $G$ есть циклы $P_{a, v_i} \cup \{ (v_i, v_j) \} \cup P_{v_j, a}$ и $P_{b, v_i} \cup \{ (v_i, v_j) \} \cup P_{v_j, b}$.

При этом, кратчайшие пути по условию существуют, то есть из этих циклов не достижимы циклы отрицательного веса, если они есть в графе $G$
(точнее, если и есть ребро из данных циклов в цикл с отрицательным весом, то обратно из этого цикла с отрицательным весом в данные циклы вернуться нельзя),
и сами циклы неотрцательного веса (иначе бы из $a$ в $a$ можно было прийти со сколько угодно малым весом пути)

Пример такого графа:

\includegraphics[scale=0.5]{a2_t4.png}

Данные ограничения на граф не гарантируют отсутствие цикла отрицательного веса, но и не накладвают дополнительные ограничения на структуру графа
кроме наличия указанных выше циклов (которые, как показано, неотрицательного веса), а 
алгоритм Дейкстры, A*, Форда-Беллмана и Флойда-Уоршелла корректно работают при наличии циклов, если их вес $\ge 0$.

\par

\pagebreak

\section*{Пункт 3.}

Рассмотрим граф $G = (V, E, w)$, где $ V = \{ 1, 2, 3, 4 \}, E = \{ (1, 2), (2, 3), (3, 4), (4, 1) \} $

Матрица весов графа:
$$
w = \begin{bmatrix}
0 & 1 & +\infty & +\infty \\
+\infty & 0 & 1 & +\infty \\
+\infty & +\infty & 0 & 1 \\
1 & +\infty & +\infty & 0 \\
\end{bmatrix}
$$

$\begin{tikzpicture}
    \node[draw,align=left] at (0, 1) {Граф G:};
    \coordinate (v1) at (0, 0);
    \coordinate (v2) at (2, 0);
    \coordinate (v3) at (2, -2);
    \coordinate (v4) at (0, -2);
    \filldraw[black] (v1) circle (1pt) node[anchor=south]{$v_1$};
    \filldraw[black] (v2) circle (1pt) node[anchor=south]{$v_2$};
    \filldraw[black] (v3) circle (1pt) node[anchor=north]{$v_3$};
    \filldraw[black] (v4) circle (1pt) node[anchor=north]{$v_4$};
    \draw[-stealth] (v1) -- node[anchor=south]{1} (v2);
    \draw[-stealth] (v2) -- node[anchor=west]{1} (v3);
    \draw[-stealth] (v3) -- node[anchor=north]{1} (v4);
    \draw[-stealth] (v4) -- node[anchor=east]{1} (v1);
\end{tikzpicture} $

Инициализация матрицы $dist[][]$
$$
dist = w = \begin{bmatrix}
0 & 1 & +\infty & +\infty \\
+\infty & 0 & 1 & +\infty \\
+\infty & +\infty & 0 & 1 \\
1 & +\infty & +\infty & 0 \\
\end{bmatrix}
$$

\raggedright
Опишем 64 итераций алгоритма в таблице ниже

(латех смещает таблицу в конец документа, вставить её между последующим выводом о получившейся матрице $dist[][]$ и фразой "Опишем 64 итераций..." не получилось)

Получившаяся матрица расстояний:

$\begin{bmatrix}
    0 & 1 & 2 & 3 \\
    +\infty & 0 & 1 & 2 \\
    2 & 3 & 0 & 1 \\
    1 & 2 & 3 & 0 \\
\end{bmatrix}$

Однако корректный алгоритм Флойда-Уоршелла завершает работу, найдя такую матрицу расстояний:

$\begin{bmatrix}
    0 & 1 & 2 & 3 \\
    3 & 0 & 1 & 2 \\
    2 & 3 & 0 & 1 \\
    1 & 2 & 3 & 0 \\
\end{bmatrix}$

Таким образом, некорректный вариант алгоритма не смог найти путь из $v_2$ в $v_1$

\par

\begin{table}[h]
\centering
\begin{tabular}{ccccccc}
\toprule
\multicolumn{1}{c}{} & \multicolumn{3}{c}{\textbf{for cycle values}} & \multicolumn{3}{c}{\textbf{dist values}} \\
\cmidrule(rl){2-4} \cmidrule(rl){5-7}
{$dist[][]$ before update} & {i} & {j} & {k} & {$dist[i][j]$ before update} & {$dist[i][k] + dist[k][j]$} & {$dist[i][j]$ after update} \\
\midrule
$\begin{bmatrix}
    0 & 1 & +\infty & +\infty \\
    +\infty & 0 & 1 & +\infty \\
    +\infty & +\infty & 0 & 1 \\
    1 & +\infty & +\infty & 0 \\
\end{bmatrix}$ & 1 & 1 & 1 & 0 & 0 & 0 \\
\bottomrule
$\begin{bmatrix}
    0 & 1 & +\infty & +\infty \\
    +\infty & 0 & 1 & +\infty \\
    +\infty & +\infty & 0 & 1 \\
    1 & +\infty & +\infty & 0 \\
\end{bmatrix}$ & 1 & 1 & 2 & 0 & $+\infty$ & 0 \\
\bottomrule
$\begin{bmatrix}
    0 & 1 & +\infty & +\infty \\
    +\infty & 0 & 1 & +\infty \\
    +\infty & +\infty & 0 & 1 \\
    1 & +\infty & +\infty & 0 \\
\end{bmatrix}$ & 1 & 1 & 3 & 0 & $+\infty$ & 0 \\
\bottomrule
$\begin{bmatrix}
    0 & 1 & +\infty & +\infty \\
    +\infty & 0 & 1 & +\infty \\
    +\infty & +\infty & 0 & 1 \\
    1 & +\infty & +\infty & 0 \\
\end{bmatrix}$ & 1 & 1 & 4 & 0 & $+\infty$ & 0 \\
\bottomrule
$\begin{bmatrix}
    0 & 1 & +\infty & +\infty \\
    +\infty & 0 & 1 & +\infty \\
    +\infty & +\infty & 0 & 1 \\
    1 & +\infty & +\infty & 0 \\
\end{bmatrix}$ & 1 & 2 & 1 & 1 & 1 & 1 \\
\bottomrule
$\begin{bmatrix}
    0 & 1 & +\infty & +\infty \\
    +\infty & 0 & 1 & +\infty \\
    +\infty & +\infty & 0 & 1 \\
    1 & +\infty & +\infty & 0 \\
\end{bmatrix}$ & 1 & 2 & 2 & 1 & 1 & 1 \\
\bottomrule
$\begin{bmatrix}
    0 & 1 & +\infty & +\infty \\
    +\infty & 0 & 1 & +\infty \\
    +\infty & +\infty & 0 & 1 \\
    1 & +\infty & +\infty & 0 \\
\end{bmatrix}$ & 1 & 2 & 3 & 1 & $+\infty$ & 1 \\
\bottomrule
$\begin{bmatrix}
    0 & 1 & +\infty & +\infty \\
    +\infty & 0 & 1 & +\infty \\
    +\infty & +\infty & 0 & 1 \\
    1 & +\infty & +\infty & 0 \\
\end{bmatrix}$ & 1 & 2 & 4 & 1 & $+\infty$ & 1 \\
\bottomrule
$\begin{bmatrix}
    0 & 1 & +\infty & +\infty \\
    +\infty & 0 & 1 & +\infty \\
    +\infty & +\infty & 0 & 1 \\
    1 & +\infty & +\infty & 0 \\
\end{bmatrix}$ & 1 & 3 & 1 & $+\infty$ & $+\infty$ & $+\infty$ \\
\bottomrule
$\begin{bmatrix}
    0 & 1 & +\infty & +\infty \\
    +\infty & 0 & 1 & +\infty \\
    +\infty & +\infty & 0 & 1 \\
    1 & +\infty & +\infty & 0 \\
\end{bmatrix}$ & 1 & 3 & 2 & $+\infty$ & 2 & 2 \\
\bottomrule
\end{tabular}
\end{table}

\begin{table}[h]
\centering
\begin{tabular}{ccccccc}
\toprule
\multicolumn{1}{c}{} & \multicolumn{3}{c}{\textbf{for cycle values}} & \multicolumn{3}{c}{\textbf{dist values}} \\
\cmidrule(rl){2-4} \cmidrule(rl){5-7}
{$dist[][]$ before update} & {i} & {j} & {k} & {$dist[i][j]$ before update} & {$dist[i][k] + dist[k][j]$} & {$dist[i][j]$ after update} \\
\midrule
$\begin{bmatrix}
    0 & 1 & 2 & +\infty \\
    +\infty & 0 & 1 & +\infty \\
    +\infty & +\infty & 0 & 1 \\
    1 & +\infty & +\infty & 0 \\
\end{bmatrix}$ & 1 & 3 & 3 & 2 & 2 & 2 \\
\bottomrule
$\begin{bmatrix}
    0 & 1 & 2 & +\infty \\
    +\infty & 0 & 1 & +\infty \\
    +\infty & +\infty & 0 & 1 \\
    1 & +\infty & +\infty & 0 \\
\end{bmatrix}$ & 1 & 3 & 4 & 2 & $+\infty$ & 2 \\
\bottomrule
$\begin{bmatrix}
    0 & 1 & 2 & +\infty \\
    +\infty & 0 & 1 & +\infty \\
    +\infty & +\infty & 0 & 1 \\
    1 & +\infty & +\infty & 0 \\
\end{bmatrix}$ & 1 & 4 & 1 & $+\infty$ & $+\infty$ & $+\infty$ \\
\bottomrule
$\begin{bmatrix}
    0 & 1 & 2 & +\infty \\
    +\infty & 0 & 1 & +\infty \\
    +\infty & +\infty & 0 & 1 \\
    1 & +\infty & +\infty & 0 \\
\end{bmatrix}$ & 1 & 4 & 2 & $+\infty$ & $+\infty$ & $+\infty$ \\
\bottomrule
$\begin{bmatrix}
    0 & 1 & 2 & +\infty \\
    +\infty & 0 & 1 & +\infty \\
    +\infty & +\infty & 0 & 1 \\
    1 & +\infty & +\infty & 0 \\
\end{bmatrix}$ & 1 & 4 & 3 & $+\infty$ & 3 & 3 \\
\bottomrule
$\begin{bmatrix}
    0 & 1 & 2 & 3 \\
    +\infty & 0 & 1 & +\infty \\
    +\infty & +\infty & 0 & 1 \\
    1 & +\infty & +\infty & 0 \\
\end{bmatrix}$ & 1 & 4 & 4 & 3 & 3 & 3 \\
\bottomrule
$\begin{bmatrix}
    0 & 1 & 2 & 3 \\
    +\infty & 0 & 1 & +\infty \\
    +\infty & +\infty & 0 & 1 \\
    1 & +\infty & +\infty & 0 \\
\end{bmatrix}$ & 2 & 1 & 1 & $+\infty$ & $+\infty$ & $+\infty$ \\
\bottomrule
$\begin{bmatrix}
    0 & 1 & 2 & 3 \\
    +\infty & 0 & 1 & +\infty \\
    +\infty & +\infty & 0 & 1 \\
    1 & +\infty & +\infty & 0 \\
\end{bmatrix}$ & 2 & 1 & 2 & $+\infty$ & $+\infty$ & $+\infty$ \\
\bottomrule
$\begin{bmatrix}
    0 & 1 & 2 & 3 \\
    +\infty & 0 & 1 & +\infty \\
    +\infty & +\infty & 0 & 1 \\
    1 & +\infty & +\infty & 0 \\
\end{bmatrix}$ & 2 & 1 & 3 & $+\infty$ & $+\infty$ & $+\infty$ \\
\bottomrule
$\begin{bmatrix}
    0 & 1 & 2 & 3 \\
    +\infty & 0 & 1 & +\infty \\
    +\infty & +\infty & 0 & 1 \\
    1 & +\infty & +\infty & 0 \\
\end{bmatrix}$ & 2 & 1 & 4 & $+\infty$ & $+\infty$ & $+\infty$ \\
\bottomrule
\end{tabular}
\end{table}

\begin{table}[h]
\centering
\begin{tabular}{ccccccc}
\toprule
\multicolumn{1}{c}{} & \multicolumn{3}{c}{\textbf{for cycle values}} & \multicolumn{3}{c}{\textbf{dist values}} \\
\cmidrule(rl){2-4} \cmidrule(rl){5-7}
{$dist[][]$ before update} & {i} & {j} & {k} & {$dist[i][j]$ before update} & {$dist[i][k] + dist[k][j]$} & {$dist[i][j]$ after update} \\
\midrule
$\begin{bmatrix}
    0 & 1 & 2 & 3 \\
    +\infty & 0 & 1 & +\infty \\
    +\infty & +\infty & 0 & 1 \\
    1 & +\infty & +\infty & 0 \\
\end{bmatrix}$ & 2 & 2 & 1 & 0 & $+\infty$ & 0 \\
\bottomrule
$\begin{bmatrix}
    0 & 1 & 2 & 3 \\
    +\infty & 0 & 1 & +\infty \\
    +\infty & +\infty & 0 & 1 \\
    1 & +\infty & +\infty & 0 \\
\end{bmatrix}$ & 2 & 2 & 2 & 0 & 0 & 0 \\
\bottomrule
$\begin{bmatrix}
    0 & 1 & 2 & 3 \\
    +\infty & 0 & 1 & +\infty \\
    +\infty & +\infty & 0 & 1 \\
    1 & +\infty & +\infty & 0 \\
\end{bmatrix}$ & 2 & 2 & 3 & 0 & $+\infty$ & 0 \\
\bottomrule
$\begin{bmatrix}
    0 & 1 & 2 & 3 \\
    +\infty & 0 & 1 & +\infty \\
    +\infty & +\infty & 0 & 1 \\
    1 & +\infty & +\infty & 0 \\
\end{bmatrix}$ & 2 & 2 & 4 & 0 & $+\infty$ & 0 \\
\bottomrule
$\begin{bmatrix}
    0 & 1 & 2 & 3 \\
    +\infty & 0 & 1 & +\infty \\
    +\infty & +\infty & 0 & 1 \\
    1 & +\infty & +\infty & 0 \\
\end{bmatrix}$ & 2 & 3 & 1 & 1 & $+\infty$ & 1 \\
\bottomrule
$\begin{bmatrix}
    0 & 1 & 2 & 3 \\
    +\infty & 0 & 1 & +\infty \\
    +\infty & +\infty & 0 & 1 \\
    1 & +\infty & +\infty & 0 \\
\end{bmatrix}$ & 2 & 3 & 2 & 1 & 1 & 1 \\
\bottomrule
$\begin{bmatrix}
    0 & 1 & 2 & 3 \\
    +\infty & 0 & 1 & +\infty \\
    +\infty & +\infty & 0 & 1 \\
    1 & +\infty & +\infty & 0 \\
\end{bmatrix}$ & 2 & 3 & 3 & 1 & 1 & 1 \\
\bottomrule
$\begin{bmatrix}
    0 & 1 & 2 & 3 \\
    +\infty & 0 & 1 & +\infty \\
    +\infty & +\infty & 0 & 1 \\
    1 & +\infty & +\infty & 0 \\
\end{bmatrix}$ & 2 & 3 & 4 & 1 & $+\infty$ & 1 \\
\bottomrule
$\begin{bmatrix}
    0 & 1 & 2 & 3 \\
    +\infty & 0 & 1 & +\infty \\
    +\infty & +\infty & 0 & 1 \\
    1 & +\infty & +\infty & 0 \\
\end{bmatrix}$ & 2 & 4 & 1 & $+\infty$ & $+\infty$ & $+\infty$ \\
\bottomrule
$\begin{bmatrix}
    0 & 1 & 2 & 3 \\
    +\infty & 0 & 1 & +\infty \\
    +\infty & +\infty & 0 & 1 \\
    1 & +\infty & +\infty & 0 \\
\end{bmatrix}$ & 2 & 4 & 2 & $+\infty$ & $+\infty$ & $+\infty$ \\
\bottomrule
\end{tabular}
\end{table}

\begin{table}[h]
\centering
\begin{tabular}{ccccccc}
\toprule
\multicolumn{1}{c}{} & \multicolumn{3}{c}{\textbf{for cycle values}} & \multicolumn{3}{c}{\textbf{dist values}} \\
\cmidrule(rl){2-4} \cmidrule(rl){5-7}
{$dist[][]$ before update} & {i} & {j} & {k} & {$dist[i][j]$ before update} & {$dist[i][k] + dist[k][j]$} & {$dist[i][j]$ after update} \\
\midrule
$\begin{bmatrix}
    0 & 1 & 2 & 3 \\
    +\infty & 0 & 1 & +\infty \\
    +\infty & +\infty & 0 & 1 \\
    1 & +\infty & +\infty & 0 \\
\end{bmatrix}$ & 2 & 4 & 3 & $+\infty$ & 2 & 2 \\
\bottomrule
$\begin{bmatrix}
    0 & 1 & 2 & 3 \\
    +\infty & 0 & 1 & 2 \\
    +\infty & +\infty & 0 & 1 \\
    1 & +\infty & +\infty & 0 \\
\end{bmatrix}$ & 2 & 4 & 4 & 2 & 2 & 2 \\
\bottomrule
$\begin{bmatrix}
    0 & 1 & 2 & 3 \\
    +\infty & 0 & 1 & 2 \\
    +\infty & +\infty & 0 & 1 \\
    1 & +\infty & +\infty & 0 \\
\end{bmatrix}$ & 3 & 1 & 1 & $+\infty$ & $+\infty$ & $+\infty$ \\
\bottomrule
$\begin{bmatrix}
    0 & 1 & 2 & 3 \\
    +\infty & 0 & 1 & 2 \\
    +\infty & +\infty & 0 & 1 \\
    1 & +\infty & +\infty & 0 \\
\end{bmatrix}$ & 3 & 1 & 2 & $+\infty$ & $+\infty$ & $+\infty$ \\
\bottomrule
$\begin{bmatrix}
    0 & 1 & 2 & 3 \\
    +\infty & 0 & 1 & 2 \\
    +\infty & +\infty & 0 & 1 \\
    1 & +\infty & +\infty & 0 \\
\end{bmatrix}$ & 3 & 1 & 3 & $+\infty$ & $+\infty$ & $+\infty$ \\
\bottomrule
$\begin{bmatrix}
    0 & 1 & 2 & 3 \\
    +\infty & 0 & 1 & 2 \\
    +\infty & +\infty & 0 & 1 \\
    1 & +\infty & +\infty & 0 \\
\end{bmatrix}$ & 3 & 1 & 4 & $+\infty$ & 2 & 2 \\
\bottomrule
$\begin{bmatrix}
    0 & 1 & 2 & 3 \\
    +\infty & 0 & 1 & 2 \\
    2 & +\infty & 0 & 1 \\
    1 & +\infty & +\infty & 0 \\
\end{bmatrix}$ & 3 & 2 & 1 & $+\infty$ & 3 & 3 \\
\bottomrule
$\begin{bmatrix}
    0 & 1 & 2 & 3 \\
    +\infty & 0 & 1 & 2 \\
    2 & 3 & 0 & 1 \\
    1 & +\infty & +\infty & 0 \\
\end{bmatrix}$ & 3 & 2 & 2 & 3 & 3 & 3 \\
\bottomrule
$\begin{bmatrix}
    0 & 1 & 2 & 3 \\
    +\infty & 0 & 1 & 2 \\
    2 & 3 & 0 & 1 \\
    1 & +\infty & +\infty & 0 \\
\end{bmatrix}$ & 3 & 2 & 3 & 3 & 3 & 3 \\
\bottomrule
$\begin{bmatrix}
    0 & 1 & 2 & 3 \\
    +\infty & 0 & 1 & 2 \\
    2 & 3 & 0 & 1 \\
    1 & +\infty & +\infty & 0 \\
\end{bmatrix}$ & 3 & 2 & 4 & 3 & $+\infty$ & 3 \\
\bottomrule
\end{tabular}
\end{table}

\begin{table}[h]
\centering
\begin{tabular}{ccccccc}
\toprule
\multicolumn{1}{c}{} & \multicolumn{3}{c}{\textbf{for cycle values}} & \multicolumn{3}{c}{\textbf{dist values}} \\
\cmidrule(rl){2-4} \cmidrule(rl){5-7}
{$dist[][]$ before update} & {i} & {j} & {k} & {$dist[i][j]$ before update} & {$dist[i][k] + dist[k][j]$} & {$dist[i][j]$ after update} \\
\midrule
$\begin{bmatrix}
    0 & 1 & 2 & 3 \\
    +\infty & 0 & 1 & 2 \\
    2 & 3 & 0 & 1 \\
    1 & +\infty & +\infty & 0 \\
\end{bmatrix}$ & 3 & 3 & 1 & 0 & 4 & 0 \\
\bottomrule
$\begin{bmatrix}
    0 & 1 & 2 & 3 \\
    +\infty & 0 & 1 & 2 \\
    2 & 3 & 0 & 1 \\
    1 & +\infty & +\infty & 0 \\
\end{bmatrix}$ & 3 & 3 & 2 & 0 & 4 & 0 \\
\bottomrule
$\begin{bmatrix}
    0 & 1 & 2 & 3 \\
    +\infty & 0 & 1 & 2 \\
    2 & 3 & 0 & 1 \\
    1 & +\infty & +\infty & 0 \\
\end{bmatrix}$ & 3 & 3 & 3 & 0 & 0 & 0 \\
\bottomrule
$\begin{bmatrix}
    0 & 1 & 2 & 3 \\
    +\infty & 0 & 1 & 2 \\
    2 & 3 & 0 & 1 \\
    1 & +\infty & +\infty & 0 \\
\end{bmatrix}$ & 3 & 3 & 4 & 0 & $+\infty$ & 0 \\
\bottomrule
$\begin{bmatrix}
    0 & 1 & 2 & 3 \\
    +\infty & 0 & 1 & 2 \\
    2 & 3 & 0 & 1 \\
    1 & +\infty & +\infty & 0 \\
\end{bmatrix}$ & 3 & 4 & 1 & 1 & 5 & 1 \\
\bottomrule
$\begin{bmatrix}
    0 & 1 & 2 & 3 \\
    +\infty & 0 & 1 & 2 \\
    2 & 3 & 0 & 1 \\
    1 & +\infty & +\infty & 0 \\
\end{bmatrix}$ & 3 & 4 & 2 & 1 & 5 & 1 \\
\bottomrule
$\begin{bmatrix}
    0 & 1 & 2 & 3 \\
    +\infty & 0 & 1 & 2 \\
    2 & 3 & 0 & 1 \\
    1 & +\infty & +\infty & 0 \\
\end{bmatrix}$ & 3 & 4 & 3 & 1 & 1 & 1 \\
\bottomrule
$\begin{bmatrix}
    0 & 1 & 2 & 3 \\
    +\infty & 0 & 1 & 2 \\
    2 & 3 & 0 & 1 \\
    1 & +\infty & +\infty & 0 \\
\end{bmatrix}$ & 3 & 4 & 4 & 1 & 1 & 1 \\
\bottomrule
$\begin{bmatrix}
    0 & 1 & 2 & 3 \\
    +\infty & 0 & 1 & 2 \\
    2 & 3 & 0 & 1 \\
    1 & +\infty & +\infty & 0 \\
\end{bmatrix}$ & 4 & 1 & 1 & 1 & 1 & 1 \\
\bottomrule
$\begin{bmatrix}
    0 & 1 & 2 & 3 \\
    +\infty & 0 & 1 & 2 \\
    2 & 3 & 0 & 1 \\
    1 & +\infty & +\infty & 0 \\
\end{bmatrix}$ & 4 & 1 & 2 & 1 & $+\infty$ & 1 \\
\bottomrule
\end{tabular}
\end{table}

\begin{table}[h]
\centering
\begin{tabular}{ccccccc}
\toprule
\multicolumn{1}{c}{} & \multicolumn{3}{c}{\textbf{for cycle values}} & \multicolumn{3}{c}{\textbf{dist values}} \\
\cmidrule(rl){2-4} \cmidrule(rl){5-7}
{$dist[][]$ before update} & {i} & {j} & {k} & {$dist[i][j]$ before update} & {$dist[i][k] + dist[k][j]$} & {$dist[i][j]$ after update} \\
\midrule
$\begin{bmatrix}
    0 & 1 & 2 & 3 \\
    +\infty & 0 & 1 & 2 \\
    2 & 3 & 0 & 1 \\
    1 & +\infty & +\infty & 0 \\
\end{bmatrix}$ & 4 & 1 & 3 & 1 & $+\infty$ & 1 \\
\bottomrule
$\begin{bmatrix}
    0 & 1 & 2 & 3 \\
    +\infty & 0 & 1 & 2 \\
    2 & 3 & 0 & 1 \\
    1 & +\infty & +\infty & 0 \\
\end{bmatrix}$ & 4 & 1 & 4 & 1 & 1 & 1 \\
\bottomrule
$\begin{bmatrix}
    0 & 1 & 2 & 3 \\
    +\infty & 0 & 1 & 2 \\
    2 & 3 & 0 & 1 \\
    1 & +\infty & +\infty & 0 \\
\end{bmatrix}$ & 4 & 2 & 1 & $+\infty$ & 2 & 2 \\
\bottomrule
$\begin{bmatrix}
    0 & 1 & 2 & 3 \\
    +\infty & 0 & 1 & 2 \\
    2 & 3 & 0 & 1 \\
    1 & 2 & +\infty & 0 \\
\end{bmatrix}$ & 4 & 2 & 2 & 2 & 2 & 2 \\
\bottomrule
$\begin{bmatrix}
    0 & 1 & 2 & 3 \\
    +\infty & 0 & 1 & 2 \\
    2 & 3 & 0 & 1 \\
    1 & 2 & +\infty & 0 \\
\end{bmatrix}$ & 4 & 2 & 3 & 2 & $+\infty$ & 2 \\
\bottomrule
$\begin{bmatrix}
    0 & 1 & 2 & 3 \\
    +\infty & 0 & 1 & 2 \\
    2 & 3 & 0 & 1 \\
    1 & 2 & +\infty & 0 \\
\end{bmatrix}$ & 4 & 2 & 4 & 2 & 2 & 2 \\
\bottomrule
$\begin{bmatrix}
    0 & 1 & 2 & 3 \\
    +\infty & 0 & 1 & 2 \\
    2 & 3 & 0 & 1 \\
    1 & 2 & +\infty & 0 \\
\end{bmatrix}$ & 4 & 3 & 1 & $+\infty$ & 3 & 3 \\
\bottomrule
$\begin{bmatrix}
    0 & 1 & 2 & 3 \\
    +\infty & 0 & 1 & 2 \\
    2 & 3 & 0 & 1 \\
    1 & 2 & 3 & 0 \\
\end{bmatrix}$ & 4 & 3 & 2 & 3 & 3 & 3 \\
\bottomrule
$\begin{bmatrix}
    0 & 1 & 2 & 3 \\
    +\infty & 0 & 1 & 2 \\
    2 & 3 & 0 & 1 \\
    1 & 2 & 3 & 0 \\
\end{bmatrix}$ & 4 & 3 & 3 & 3 & 3 & 3 \\
\bottomrule
$\begin{bmatrix}
    0 & 1 & 2 & 3 \\
    +\infty & 0 & 1 & 2 \\
    2 & 3 & 0 & 1 \\
    1 & 2 & 3 & 0 \\
\end{bmatrix}$ & 4 & 3 & 4 & 3 & 3 & 3 \\
\bottomrule
\end{tabular}
\end{table}

\begin{table}[h]
\centering
\begin{tabular}{ccccccc}
\toprule
\multicolumn{1}{c}{} & \multicolumn{3}{c}{\textbf{for cycle values}} & \multicolumn{3}{c}{\textbf{dist values}} \\
\cmidrule(rl){2-4} \cmidrule(rl){5-7}
{$dist[][]$ before update} & {i} & {j} & {k} & {$dist[i][j]$ before update} & {$dist[i][k] + dist[k][j]$} & {$dist[i][j]$ after update} \\
\midrule
$\begin{bmatrix}
    0 & 1 & 2 & 3 \\
    +\infty & 0 & 1 & 2 \\
    2 & 3 & 0 & 1 \\
    1 & 2 & 3 & 0 \\
\end{bmatrix}$ & 4 & 4 & 1 & 0 & 4 & 0 \\
\bottomrule
$\begin{bmatrix}
    0 & 1 & 2 & 3 \\
    +\infty & 0 & 1 & 2 \\
    2 & 3 & 0 & 1 \\
    1 & 2 & 3 & 0 \\
\end{bmatrix}$ & 4 & 4 & 2 & 0 & 4 & 0 \\
\bottomrule
$\begin{bmatrix}
    0 & 1 & 2 & 3 \\
    +\infty & 0 & 1 & 2 \\
    2 & 3 & 0 & 1 \\
    1 & 2 & 3 & 0 \\
\end{bmatrix}$ & 4 & 4 & 3 & 0 & 4 & 0 \\
\bottomrule
$\begin{bmatrix}
    0 & 1 & 2 & 3 \\
    +\infty & 0 & 1 & 2 \\
    2 & 3 & 0 & 1 \\
    1 & 2 & 3 & 0 \\
\end{bmatrix}$ & 4 & 4 & 4 & 0 & 0 & 0 \\
\bottomrule
\end{tabular}
\end{table}

\end{document}
