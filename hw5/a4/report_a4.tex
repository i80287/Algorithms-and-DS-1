\documentclass{report}
% \includeonly{pars/9_continuous_functions_and_limits, parts/10_integration_of_functions}
% \documentclass[11pt,a4paper]{article}

%%%%%%%%%%%%%%%%%%%%%%%%%%%%%%%%%
% PACKAGE IMPORTS
%%%%%%%%%%%%%%%%%%%%%%%%%%%%%%%%%

\usepackage[utf8]{inputenc} % Сурс - семинары по алгебре Медведя Никиты Юрьевича
\usepackage[T2A]{fontenc}

\usepackage[tmargin=2cm,rmargin=1in,lmargin=1in,margin=0.85in,bmargin=2cm,footskip=.2in]{geometry}
\usepackage{amsmath,amsfonts,amsthm,amssymb,mathtools}
\usepackage[varbb]{newpxmath}
\usepackage{xfrac}
\usepackage[makeroom]{cancel}
\usepackage{mathtools}
\usepackage{bookmark}
\usepackage{enumitem}
\usepackage{hyperref,theoremref}
\hypersetup{
	pdftitle={Assignment},
	colorlinks=true, linkcolor=doc!90,
	bookmarksnumbered=true,
	bookmarksopen=true
}
\usepackage[most,many,breakable]{tcolorbox}
\usepackage{xcolor}% http://ctan.org/pkg/xcolor
%\usepackage{colortbl}% http://ctan.org/pkg/colortbl
\usepackage{multirow}% http://ctan.org/pkg/multirow
\usepackage{graphicx}% http://ctan.org/pkg/graphicx
\usepackage{varwidth}
\usepackage{varwidth}
\usepackage{etoolbox}
%\usepackage{authblk}
\usepackage{nameref}
\usepackage{multicol,array}
\usepackage{tikz-cd}
\usepackage[ruled,vlined,linesnumbered]{algorithm2e}
\usepackage{comment} % enables the use of multi-line comments (\ifx \fi) 
\usepackage{import}
\usepackage{xifthen}
\usepackage{pdfpages}
\usepackage{transparent}

\usepackage[english]{babel}
% ,russian
\usepackage{amsfonts,amssymb}
\usepackage{relsize}

\usepackage{systeme}

\usepackage{indentfirst} % Красная строка
\usepackage{fancyhdr}
\usepackage{wrapfig}
\usepackage{textcomp}

% \usepackage[unicode]{hyperref}

\newcommand\mycommfont[1]{\footnotesize\ttfamily\textcolor{blue}{#1}}
\SetCommentSty{mycommfont}
\newcommand{\incfig}[1]{%
    \def\svgwidth{\columnwidth}
    \import{./figures/}{#1.pdf_tex}
}

\usepackage{tikzsymbols}
\renewcommand\qedsymbol{$\Laughey$}


%\usepackage{import}
%\usepackage{xifthen}
%\usepackage{pdfpages}
%\usepackage{transparent}


%%%%%%%%%%%%%%%%%%%%%%%%%%%%%%
% SELF MADE COLORS
%%%%%%%%%%%%%%%%%%%%%%%%%%%%%%



\definecolor{myg}{RGB}{56, 140, 70}
\definecolor{myb}{RGB}{45, 111, 177}
\definecolor{myr}{RGB}{199, 68, 64}
\definecolor{mytheorembg}{HTML}{F2F2F9}
\definecolor{mytheoremfr}{HTML}{00007B}
\definecolor{mylenmabg}{HTML}{FFFAF8}
\definecolor{mylenmafr}{HTML}{983b0f}
\definecolor{mypropbg}{HTML}{f2fbfc}
\definecolor{mypropfr}{HTML}{191971}
\definecolor{myexamplebg}{HTML}{F2FBF8}
\definecolor{myexamplefr}{HTML}{88D6D1}
\definecolor{myexampleti}{HTML}{2A7F7F}
\definecolor{mydefinitbg}{HTML}{E5E5FF}
\definecolor{mydefinitfr}{HTML}{3F3FA3}
\definecolor{notesgreen}{RGB}{0,162,0}
\definecolor{myp}{RGB}{197, 92, 212}
\definecolor{mygr}{HTML}{2C3338}
\definecolor{myred}{RGB}{127,0,0}
\definecolor{myyellow}{RGB}{169,121,69}
\definecolor{myexercisebg}{HTML}{F2FBF8}
\definecolor{myexercisefg}{HTML}{88D6D1}

%%%%%%%%%%%%%%%%%%%%%%%%%%%%%%%%%%%%%%%%%%%
% TABLE OF CONTENTS
%%%%%%%%%%%%%%%%%%%%%%%%%%%%%%%%%%%%%%%%%%%

\usepackage{tikz}
\definecolor{doc}{RGB}{0,60,110}
\usepackage{titletoc}
\contentsmargin{0cm}
\titlecontents{chapter}[3.7pc]
{\addvspace{30pt}%
	\begin{tikzpicture}[remember picture, overlay]%
		\draw[fill=doc!60,draw=doc!60] (-7,-.1) rectangle (-0.9,.5);%
		\pgftext[left,x=-3.5cm,y=0.2cm]{\color{white}\Large\sc\bfseries Chapter\ \thecontentslabel};%
	\end{tikzpicture}\color{doc!60}\large\sc\bfseries}%
{}
{}
{\;\titlerule\;\large\sc\bfseries Page \thecontentspage
	\begin{tikzpicture}[remember picture, overlay]
		\draw[fill=doc!60,draw=doc!60] (2pt,0) rectangle (4,0.1pt);
	\end{tikzpicture}}%
\titlecontents{section}[3.7pc]
{\addvspace{2pt}}
{\contentslabel[\thecontentslabel]{2pc}}
{}
{\hfill\small \thecontentspage}
[]
\titlecontents*{subsection}[3.7pc]
{\addvspace{-1pt}\small}
{}
{}
{\ --- \small\thecontentspage}
[ \textbullet\ ][]

\makeatletter
\renewcommand{\tableofcontents}{%
	\chapter*{%
	  \vspace*{-20\p@}%
	  \begin{tikzpicture}[remember picture, overlay]%
		  \pgftext[right,x=15cm,y=0.2cm]{\color{doc!60}\Huge\sc\bfseries \contentsname};%
		  \draw[fill=doc!60,draw=doc!60] (13,-.75) rectangle (20,1);%
		  \clip (13,-.75) rectangle (20,1);
		  \pgftext[right,x=15cm,y=0.2cm]{\color{white}\Huge\sc\bfseries \contentsname};%
	  \end{tikzpicture}}%
	\@starttoc{toc}}
\makeatother


\input{macros}

\DeclareMathOperator{\Hash}{Hash}
\DeclareMathOperator{\hash}{hash}

\begin{document}

Пусть $ F $ - данный фильтр Блума

\mcdfn{Область определения фильтра Блума}{
    Через $D_{F}$ обозначим область определения фильтра Блума $F$, 
    т.е. множество значений, которые можно добавить в фильтр
}

\mcclm{Принадлежность фильтру Блума}{}{
    Пусть задан некоторый фильтр Блума $ F(A) $ на множестве $ A \subseteq D_{F} $

    Введём обозначение: $ \forall x: x \in F(A) \iff $ фильтр выдал ответ о принадлежности объекта $x$ фильтру
}

\nt{
    По построению фильтра Блума $F$:

    $ \forall A \subseteq D_{F} \, \forall x \left( x \in A \implies x \in F(A) \right) $
}

Т.к. в условии задачи не сказано обратное, будем считать, что для всех фильтров 
Блума $F(A), A \subseteq D_F$ выбрано одинаковое количество хэш функций так, 
что в одном фильтре функции могут быть не равны, но для любых двух фильтров 
последовательность их хэш функций совпадает.

(если у каждого фильтра свои хэш функции, то в общем случае ответ на оба вопроса, очевидно, нет)

Иначе говоря, нам дан один фильтр Блума, реализации которого для различных множеств $A$ будут различаться,
но правила построения (т.е. длина массива и последовательность хэш функций) всегда одинаковые.

\mcclm{Дополнительные обозначения}{}{
    Для всех рассматриваемых фильтров Блума через $n$ обозначим длину их битового массива, а через
    $m$ обозначим количество хэш функций. Последовательность значений хэш функций по модулю $n$ 
    обозначим $ (h_1, h_2, ..., h_m) $, то есть 
        $ \forall i \in \{ 1, 2, ..., m \}: \left( h_i: D_{F} \to \{ 0, 1, ..., n - 1 \} \right) $
}

\nt{
    $ \forall x \in D_F$ значения $ h_1(x), h_2(x), ..., h_m(x) $ не обязательно попарно различны
}

\qs{}{
    Верно ли, что $F(AB)$ будет выдавать положительные ответы о принадлежности объектов из множества $ A \cap B$ ? Почему (нет)?

    Докажем, что утверждение верно, т.е. ответ - да.
\mcprf{
\begin{split}
    & \text{1. По условию даны фильтр Блума $F(A)$, построенный на множестве $A \subseteq D_F $,} \\
    & \text{и фильтр Блума $F(B)$, построенный на множестве $B \subseteq D_F $.} \\
    & \text{Через $F(AB)$ обозначен фильтр c битовым массивом, полученным путём побитового И }\\
    & \text{над битовыми массивами фильтров $F(A)$ и $F(B)$} \\
    & \text{2. Пусть $x \in A \cap B$, тогда, т.к. в фильтрах $F(A)$ и $F(B)$ используются одинаковые хэш-функции,} \\
    & \text{то и в битовом массиве фильтра $F(A)$, и в битовом массиве фильтра $F(B)$ на позициях }\\
    & h_1(x), h_2(x), ..., h_m(x) \text{ будет стоять 1, тогда после применения побитового И к массивам} \\
    & \text{в получившемся битовом массиве на этих позициях будет стоять 1 $\implies x \in F(AB)$ } \\
\end{split}
}
}

\pagebreak

\qs{}{
    Верно ли, что $F(AB)$ будет в точности соответствовать другому фильтру, который будет получен 
    в результате последовательной вставки объектов из множества $ A \cap B$? Почему (нет)?

    Докажем, что утверждение неверно, т.е. ответ - нет, приведя контрпример:

\[
\begin{split}
    & \text{1. Рассмотрим фильтр Блума, в котором } n = 3, m = 2 \\
    & \text{Рассмотри 3 попарно различных объекта $ x, y, z $, положим $ D_F = \{ x, y, z \}, A = \{ x \}, B \{ y \} $} \\
    & \text{тогда $ A \cap B = \varnothing $} \\
    & \text{2. Определим 2 хэш функции (по модулю $n$):} \\
    & h_1: D_F \to \{ 0, 1, 2 \} \\
    & h_1(x) = 0 \\
    & h_1(y) = 1 \\
    & h_1(z) = 1 \\
    & h_2: D_F \to \{ 0, 1, 2 \} \\
    & h_2(x) = 1 \\
    & h_2(y) = 2 \\
    & h_2(z) = 1 \\
    & \text{3. Тогда битовый массив в $F(A)$ - это кортеж из 3 битов } (1, 1, 0) \\
    & \text{Битовый массив в $F(B)$ - это кортеж из 3 битов } (0, 1, 1) \\
    & \text{Битовый массив в $F(AB)$ - это кортеж из 3 битов } (1, 1, 0) \& (0, 1, 1) = (0, 1, 0) \\
    & \text{Битовый массив в $F(A \cap B) = F(\varnothing)$ - это кортеж из 3 битов } (0, 0, 0) \\
    & \text{Получилось, что в $F(AB)$ и $F(A \cap B)$ битовые массивы не совпадают.} \\
    & \text{В частности, $ z \in F(AB) \wedge z \notin F(A \cap B) $} \\
\end{split}
\]
}

\end{document}
