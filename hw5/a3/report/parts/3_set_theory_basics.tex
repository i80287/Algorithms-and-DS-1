\chapter{Элементы теории множеств}

\section{Аксиома непрерывности}

\mcclm{Аксиома непрерывности действительных чисел (принцип полноты)}{}
{
$$
\left. \begin{tabular}{l}
    $ A \subseteq \Rset $ \\
    $ A \ne \varnothing $ \\
    $ B \subseteq \Rset $ \\
    $ B \ne \varnothing $ \\
    $ \forall a \in A \, \forall b \in B: a \le b $
\end{tabular} \right\}
\begin{tabular}{l}
    $ \implies \exists c \in \Rset \, \forall a \in A \, \forall b \in B: a \le c \le b $
\end{tabular}
$$
}

\section{Определения ограниченных множеств}

\mcdfn{Ограниченное сверху множество}
{
    Подможество $ A \subseteq \Rset $ называется ограниченным
    свеху, если
    $ \exists C \in \Rset \,
        \forall a \in A: \, a \le C $
}

\mcdfn{Ограниченное снизу множество}
{
    Подможество $ A \subseteq \Rset $ называется ограниченным
    снизу, если
    $ \exists C \in \Rset \,
        \forall a \in A: \, a \ge C $
}

\mcdfn{Ограниченное множество}
{
    Подможество $ A \subseteq \Rset $ называется ограниченным, если
    $ \exists C > 0 \,
        \forall a \in A: \, | a | \le C $
}

\section{Определения граней множества}

\mcdfn{Определение верхней грани множества}{
    Пусть дано множество $ A \subset \Rset \wedge A \ne \varnothing $.
    Тогда верхней гранью множества $ A $ называют число $ c \in \Rset $, такое что
    $ \forall a \in A: a \le c $
}

\mcdfn{Определение нижней грани множества}{
    Пусть дано множество $ A \subset \Rset \wedge A \ne \varnothing $.
    Тогда нижней гранью множества $ A $ называют число $ c \in \Rset $, такое что
    $ \forall a \in A: a \ge c $
}

\mcdfn{Определение точной верхней грани множества}{
    Пусть дано множество $ A \subset \Rset \wedge A \ne \varnothing $.
    Тогда точной верхней гранью множества $ A $ называют наименьший элемента множества
    всех верхних граней множества $ A $ и обозначают $ \sup A $
}

\mcdfn{Определение точной нижней грани множества}{
    Пусть дано множество $ A \subset \Rset \wedge A \ne \varnothing $.
    Тогда точной нижней гранью множества $ A $ называют наибольший элемента множества
    всех нижней граней множества $ A $ и обозначают $ \inf A $
}

\nt{
    Вообще говоря, наименьшый и наибольший элементы множества не всегда существуют. \\
    Например, у множества $(0; 1)$ нет ни наименьшего, ни наибольшего элементов, при этом \\
    $ \sup (0; 1) = 1 \notin (0; 1) $, $ \inf (0; 1) = 0 \notin (0; 1) $
}

\section{Теорема о существовании точной грани множества}

\mcthm{Теорема о существовании точной грани множества}
{
Если множество $ A \subset \Rset, A \ne \varnothing $ ограничено сверху, то $ \exists \sup A $

Если множество $ A \subset \Rset, A \ne \varnothing $ ограничено снизу, то $ \exists \inf A $

\begin{mcproof}
    Докажем для верхней грани, для нижней грани доказательство аналогично
    \[ A \subseteq \Rset
        \wedge A \ne \varnothing
        \wedge (\exists C > 0 \, \forall a \in A
            \implies a < C)
                \implies \exists \sup{A} \]
\begin{equation*}
\begin{split}
    & \text{1. Обозначим } S_A = \{ c \in \Rset | \forall a \in A \implies a \le c \} \ne \varnothing \text{ - множество верхних граней} \\
    & \text{Это множество не пусто, т.к. $ A $ ограничено по условию, т.е. } \exists c > 0 \, \forall a \in A \implies a \le c \\
    & \text{2. По построению множества $ A $ и $ S_A $ удовлетворяют аксиоме непрерывности } \\
    & \text{действительных чисел, тогда } \exists b \in \Rset \, \forall a \in A \forall c \in S_A \implies a \le b \le c \\
    & \text{Но из $ b \le c \implies b \in S_A $, при этом } (\forall c \in S_A \implies b \le c) \text{, следовательно, $ b $ является } \\
    & \text{наименьшим элементом множества верхних граней множества $ A $, тогда по определению } \\
    & \text{точной верхней грани } b = \sup{A} \\
\end{split}
\end{equation*}    
\end{mcproof}
}
