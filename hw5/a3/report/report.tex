\documentclass{report}
% \includeonly{pars/9_continuous_functions_and_limits, parts/10_integration_of_functions}
% \documentclass[11pt,a4paper]{article}

%%%%%%%%%%%%%%%%%%%%%%%%%%%%%%%%%
% PACKAGE IMPORTS
%%%%%%%%%%%%%%%%%%%%%%%%%%%%%%%%%

\usepackage[utf8]{inputenc} % Сурс - семинары по алгебре Медведя Никиты Юрьевича
\usepackage[T2A]{fontenc}

\usepackage[tmargin=2cm,rmargin=1in,lmargin=1in,margin=0.85in,bmargin=2cm,footskip=.2in]{geometry}
\usepackage{amsmath,amsfonts,amsthm,amssymb,mathtools}
\usepackage[varbb]{newpxmath}
\usepackage{xfrac}
\usepackage[makeroom]{cancel}
\usepackage{mathtools}
\usepackage{bookmark}
\usepackage{enumitem}
\usepackage{hyperref,theoremref}
\hypersetup{
	pdftitle={Assignment},
	colorlinks=true, linkcolor=doc!90,
	bookmarksnumbered=true,
	bookmarksopen=true
}
\usepackage[most,many,breakable]{tcolorbox}
\usepackage{xcolor}% http://ctan.org/pkg/xcolor
%\usepackage{colortbl}% http://ctan.org/pkg/colortbl
\usepackage{multirow}% http://ctan.org/pkg/multirow
\usepackage{graphicx}% http://ctan.org/pkg/graphicx
\usepackage{varwidth}
\usepackage{varwidth}
\usepackage{etoolbox}
\usepackage{anyfontsize}
% \usepackage{lmodern}
%\usepackage{authblk}
\usepackage{nameref}
\usepackage{multicol,array}
\usepackage{tikz-cd}
\usepackage[ruled,vlined,linesnumbered]{algorithm2e}
\usepackage{comment} % enables the use of multi-line comments (\ifx \fi) 
\usepackage{import}
\usepackage{xifthen}
\usepackage{pdfpages}
\usepackage{transparent}

\usepackage[english]{babel}
% ,russian
\usepackage{amsfonts,amssymb}
\usepackage{relsize}

\usepackage{systeme}

\usepackage{indentfirst} % Красная строка
\usepackage{fancyhdr}
\usepackage{wrapfig}
\usepackage{textcomp}

\usepackage{physics}

% \usepackage[unicode]{hyperref}

\newcommand\mycommfont[1]{\footnotesize\ttfamily\textcolor{blue}{#1}}
\SetCommentSty{mycommfont}
\newcommand{\incfig}[1]{%
    \def\svgwidth{\columnwidth}
    \import{./figures/}{#1.pdf_tex}
}

\usepackage{tikzsymbols}
\renewcommand\qedsymbol{$\blacksquare$}

\usepackage{sectsty}

% strikethrough text \sout
\usepackage[normalem]{ulem}

\usepackage{code}

\chapternumberfont{\normalsize} 
\chaptertitlefont{\Large}
% \fontsize{<size>}{<line space>}
\sectionfont{\fontsize{12}{15}\selectfont}

% <chapter>.<section> -> <section> indexing in report mode
% \renewcommand{\thesection}{\arabic{section}}

%\usepackage{import}
%\usepackage{xifthen}
%\usepackage{pdfpages}
%\usepackage{transparent}


%%%%%%%%%%%%%%%%%%%%%%%%%%%%%%
% SELF MADE COLORS
%%%%%%%%%%%%%%%%%%%%%%%%%%%%%%



\definecolor{myg}{RGB}{56, 140, 70}
\definecolor{myb}{RGB}{45, 111, 177}
\definecolor{myr}{RGB}{199, 68, 64}
\definecolor{mytheorembg}{HTML}{F2F2F9}
\definecolor{mytheoremfr}{HTML}{00007B}
\definecolor{mylemmabg}{HTML}{FFFAF8}
\definecolor{mylemmafr}{HTML}{983b0f}
\definecolor{mypropbg}{HTML}{f2fbfc}
\definecolor{mypropfr}{HTML}{191971}
\definecolor{myexamplebg}{HTML}{F2FBF8}
\definecolor{myexamplefr}{HTML}{88D6D1}
\definecolor{myexampleti}{HTML}{2A7F7F}
\definecolor{mydefinitbg}{HTML}{E5E5FF}
\definecolor{mydefinitfr}{HTML}{3F3FA3}
\definecolor{notesgreen}{RGB}{0,162,0}
\definecolor{myp}{RGB}{197, 92, 212}
\definecolor{mygr}{HTML}{2C3338}
\definecolor{myred}{RGB}{127,0,0}
\definecolor{myyellow}{RGB}{169,121,69}
\definecolor{myexercisebg}{HTML}{F2FBF8}
\definecolor{myexercisefg}{HTML}{88D6D1}

\definecolor{myclarificationbg}{HTML}{FFFAF8}
\definecolor{myclarificationfr}{HTML}{983b0f}


%%%%%%%%%%%%%%%%%%%%%%%%%%%%
% TCOLORBOX SETUPS
%%%%%%%%%%%%%%%%%%%%%%%%%%%%

\setlength{\parindent}{1cm}
%================================
% THEOREM BOX
%================================

\tcbuselibrary{theorems,skins,hooks}
\newtcbtheorem[number within=section]{Theorem}{Theorem}
{%
	enhanced,
	breakable,
	colback = mytheorembg,
	frame hidden,
	boxrule = 0sp,
	borderline west = {2pt}{0pt}{mytheoremfr},
	sharp corners,
	detach title,
	before upper = \tcbtitle\par\smallskip,
	coltitle = mytheoremfr,
	fonttitle = \bfseries\sffamily,
	description font = \mdseries,
	separator sign none,
	segmentation style={solid, mytheoremfr},
}
{th}

\tcbuselibrary{theorems,skins,hooks}
\newtcbtheorem[number within=chapter]{theorem}{Theorem}
{%
	enhanced,
	% breakable,
	colback = mytheorembg,
	frame hidden,
	boxrule = 0sp,
	borderline west = {2pt}{0pt}{mytheoremfr},
	sharp corners,
	detach title,
	before upper = \tcbtitle\par\smallskip,
	coltitle = mytheoremfr,
	fonttitle = \bfseries\sffamily,
	description font = \mdseries,
	separator sign none,
	segmentation style={solid, mytheoremfr},
}
{th}


\tcbuselibrary{theorems,skins,hooks}
\newtcolorbox{Theoremcon}
{%
	enhanced
	,breakable
	,colback = mytheorembg
	,frame hidden
	,boxrule = 0sp
	,borderline west = {2pt}{0pt}{mytheoremfr}
	,sharp corners
	,description font = \mdseries
	,separator sign none
}

%================================
% Corollery
%================================
\tcbuselibrary{theorems,skins,hooks}
\newtcbtheorem[number within=section]{Corollary}{Corollary}
{%
	enhanced
	,breakable
	,colback = myp!10
	,frame hidden
	,boxrule = 0sp
	,borderline west = {2pt}{0pt}{myp!85!black}
	,sharp corners
	,detach title
	,before upper = \tcbtitle\par\smallskip
	,coltitle = myp!85!black
	,fonttitle = \bfseries\sffamily
	,description font = \mdseries
	,separator sign none
	,segmentation style={solid, myp!85!black}
}
{th}
\tcbuselibrary{theorems,skins,hooks}
\newtcbtheorem[number within=chapter]{corollary}{Corollary}
{%
	enhanced,
	% breakable,
	colback = myp!10,
	frame hidden,
	boxrule = 0sp,
	borderline west = {2pt}{0pt}{myp!85!black},
	sharp corners,
	detach title,
	before upper = \tcbtitle\par\smallskip,
	coltitle = myp!85!black,
	fonttitle = \bfseries\sffamily,
	description font = \mdseries,
	separator sign none,
	segmentation style={solid, myp!85!black}
}
{th}


%================================
% LENMA
%================================

\tcbuselibrary{theorems,skins,hooks}
\newtcbtheorem[number within=section]{Lemma}{Lemma}
{%
	enhanced,
	breakable,
	colback = mylemmabg,
	frame hidden,
	boxrule = 0sp,
	borderline west = {2pt}{0pt}{mylemmafr},
	sharp corners,
	detach title,
	before upper = \tcbtitle\par\smallskip,
	coltitle = mylemmafr,
	fonttitle = \bfseries\sffamily,
	description font = \mdseries,
	separator sign none,
	segmentation style={solid, mylemmafr},
}
{th}

\tcbuselibrary{theorems,skins,hooks}
\newtcbtheorem[number within=chapter]{lemma}{Lemma}
{%
	enhanced,
	% breakable,
	colback = mylemmabg,
	frame hidden,
	boxrule = 0sp,
	borderline west = {2pt}{0pt}{mylemmafr},
	sharp corners,
	detach title,
	before upper = \tcbtitle\par\smallskip,
	coltitle = mylemmafr,
	fonttitle = \bfseries\sffamily,
	description font = \mdseries,
	separator sign none,
	segmentation style={solid, mylemmafr},
}
{th}


%================================
% PROPOSITION
%================================

\tcbuselibrary{theorems,skins,hooks}
\newtcbtheorem[number within=section]{Prop}{Proposition}
{%
	enhanced,
	breakable,
	colback = mypropbg,
	frame hidden,
	boxrule = 0sp,
	borderline west = {2pt}{0pt}{mypropfr},
	sharp corners,
	detach title,
	before upper = \tcbtitle\par\smallskip,
	coltitle = mypropfr,
	fonttitle = \bfseries\sffamily,
	description font = \mdseries,
	separator sign none,
	segmentation style={solid, mypropfr},
}
{th}

\tcbuselibrary{theorems,skins,hooks}
\newtcbtheorem[number within=chapter]{prop}{Proposition}
{%
	enhanced,
	% breakable,
	colback = mypropbg,
	frame hidden,
	boxrule = 0sp,
	borderline west = {2pt}{0pt}{mypropfr},
	sharp corners,
	detach title,
	before upper = \tcbtitle\par\smallskip,
	coltitle = mypropfr,
	fonttitle = \bfseries\sffamily,
	description font = \mdseries,
	separator sign none,
	segmentation style={solid, mypropfr},
}
{th}


%================================
% CLAIM
%================================

\tcbuselibrary{theorems,skins,hooks}
\newtcbtheorem[number within=section]{Claim}{Claim}
{%
	enhanced,
	breakable,
	colback = myg!10,
	frame hidden,
	boxrule = 0sp,
	borderline west = {2pt}{0pt}{myg},
	sharp corners,
	detach title,
	before upper = \tcbtitle\par\smallskip,
	coltitle = myg!85!black,
	fonttitle = \bfseries\sffamily,
	description font = \mdseries,
	separator sign none,
	segmentation style={solid, myg!85!black}
}
{th}

\tcbuselibrary{theorems,skins,hooks}
\newtcbtheorem[number within=section]{claim}{Claim}
{%
	enhanced,
	% breakable,
	colback = myg!10,
	frame hidden,
	boxrule = 0sp,
	borderline west = {2pt}{0pt}{myg},
	sharp corners,
	detach title,
	before upper = \tcbtitle\par\smallskip,
	coltitle = myg!85!black,
	fonttitle = \bfseries\sffamily,
	description font = \mdseries,
	separator sign none,
	segmentation style={solid, myg!85!black}
}
{th}


%================================
% Exercise
%================================

\tcbuselibrary{theorems,skins,hooks}
\newtcbtheorem[number within=section]{Exercise}{Exercise}
{%
	enhanced,
	breakable,
	colback = myexercisebg,
	frame hidden,
	boxrule = 0sp,
	borderline west = {2pt}{0pt}{myexercisefg},
	sharp corners,
	detach title,
	before upper = \tcbtitle\par\smallskip,
	coltitle = myexercisefg,
	fonttitle = \bfseries\sffamily,
	description font = \mdseries,
	separator sign none,
	segmentation style={solid, myexercisefg},
}
{th}

\tcbuselibrary{theorems,skins,hooks}
\newtcbtheorem[number within=chapter]{exercise}{Exercise}
{%
	enhanced,
	% breakable,
	colback = myexercisebg,
	frame hidden,
	boxrule = 0sp,
	borderline west = {2pt}{0pt}{myexercisefg},
	sharp corners,
	detach title,
	before upper = \tcbtitle\par\smallskip,
	coltitle = myexercisefg,
	fonttitle = \bfseries\sffamily,
	description font = \mdseries,
	separator sign none,
	segmentation style={solid, myexercisefg},
}
{th}

%================================
% EXAMPLE BOX
%================================

\newtcbtheorem[number within=section]{Example}{Example}
{%
	colback = myexamplebg,
	breakable,
	colframe = myexamplefr,
	coltitle = myexampleti,
	boxrule = 1pt,
	sharp corners,
	detach title,
	before upper=\tcbtitle\par\smallskip,
	fonttitle = \bfseries,
	description font = \mdseries,
	separator sign none,
	description delimiters parenthesis
}
{ex}

\newtcbtheorem[number within=chapter]{example}{Example}
{%
	colback = myexamplebg,
	% breakable,
	colframe = myexamplefr,
	coltitle = myexampleti,
	boxrule = 1pt,
	sharp corners,
	detach title,
	before upper=\tcbtitle\par\smallskip,
	fonttitle = \bfseries,
	description font = \mdseries,
	separator sign none,
	description delimiters parenthesis
}
{ex}

%================================
% DEFINITION BOX
%================================

\newtcbtheorem[number within=section]{Definition}{Definition}{enhanced,
	before skip=2mm,after skip=2mm, colback=red!5,colframe=red!80!black,boxrule=0.5mm,
	attach boxed title to top left={xshift=1cm,yshift*=1mm-\tcboxedtitleheight}, varwidth boxed title*=-3cm,
	boxed title style={frame code={
					\path[fill=tcbcolback]
					([yshift=-1mm,xshift=-1mm]frame.north west)
					arc[start angle=0,end angle=180,radius=1mm]
					([yshift=-1mm,xshift=1mm]frame.north east)
					arc[start angle=180,end angle=0,radius=1mm];
					\path[left color=tcbcolback!60!black,right color=tcbcolback!60!black,
						middle color=tcbcolback!80!black]
					([xshift=-2mm]frame.north west) -- ([xshift=2mm]frame.north east)
					[rounded corners=1mm]-- ([xshift=1mm,yshift=-1mm]frame.north east)
					-- (frame.south east) -- (frame.south west)
					-- ([xshift=-1mm,yshift=-1mm]frame.north west)
					[sharp corners]-- cycle;
				},interior engine=empty,
		},
	fonttitle=\bfseries,
	title={#2},#1}{def}
\newtcbtheorem[number within=chapter]{definition}{Definition}{enhanced,
	before skip=2mm,after skip=2mm, colback=red!5,colframe=red!80!black,boxrule=0.5mm,
	attach boxed title to top left={xshift=1cm,yshift*=1mm-\tcboxedtitleheight}, varwidth boxed title*=-3cm,
	boxed title style={frame code={
					\path[fill=tcbcolback]
					([yshift=-1mm,xshift=-1mm]frame.north west)
					arc[start angle=0,end angle=180,radius=1mm]
					([yshift=-1mm,xshift=1mm]frame.north east)
					arc[start angle=180,end angle=0,radius=1mm];
					\path[left color=tcbcolback!60!black,right color=tcbcolback!60!black,
						middle color=tcbcolback!80!black]
					([xshift=-2mm]frame.north west) -- ([xshift=2mm]frame.north east)
					[rounded corners=1mm]-- ([xshift=1mm,yshift=-1mm]frame.north east)
					-- (frame.south east) -- (frame.south west)
					-- ([xshift=-1mm,yshift=-1mm]frame.north west)
					[sharp corners]-- cycle;
				},interior engine=empty,
		},
	fonttitle=\bfseries,
	title={#2},#1}{def}



%================================
% Solution BOX
%================================

\makeatletter
\newtcolorbox{solution}{enhanced,
	breakable,
	colback=white,
	colframe=myg!80!black,
	attach boxed title to top left={yshift*=-\tcboxedtitleheight},
	title=Solution,
	boxed title size=title,
	boxed title style={%
			sharp corners,
			rounded corners=northwest,
			colback=tcbcolframe,
			boxrule=0pt,
		},
	underlay boxed title={%
			\path[fill=tcbcolframe] (title.south west)--(title.south east)
			to[out=0, in=180] ([xshift=5mm]title.east)--
			(title.center-|frame.east)
			[rounded corners=\kvtcb@arc] |-
			(frame.north) -| cycle;
		},
}
\makeatother

%================================
% Question BOX
%================================

\makeatletter
\newtcbtheorem{Question}{Question}{enhanced,
	breakable,
	colback=white,
	colframe=mygr,
	attach boxed title to top left={yshift*=-\tcboxedtitleheight},
	fonttitle=\bfseries,
	title={#2},
	boxed title size=title,
	boxed title style={%
			sharp corners,
			rounded corners=northwest,
			colback=tcbcolframe,
			boxrule=0pt,
		},
	underlay boxed title={%
			\path[fill=tcbcolframe] (title.south west)--(title.south east)
			to[out=0, in=180] ([xshift=5mm]title.east)--
			(title.center-|frame.east)
			[rounded corners=\kvtcb@arc] |-
			(frame.north) -| cycle;
		},
	#1
}{def}
\makeatother

\makeatletter
\newtcbtheorem{question}{Question}{enhanced,
	% breakable,
	colback=white,
	colframe=mygr,
	attach boxed title to top left={yshift*=-\tcboxedtitleheight},
	fonttitle=\bfseries,
	title={#2},
	boxed title size=title,
	boxed title style={%
			sharp corners,
			rounded corners=northwest,
			colback=tcbcolframe,
			boxrule=0pt,
		},
	underlay boxed title={%
			\path[fill=tcbcolframe] (title.south west)--(title.south east)
			to[out=0, in=180] ([xshift=5mm]title.east)--
			(title.center-|frame.east)
			[rounded corners=\kvtcb@arc] |-
			(frame.north) -| cycle;
		},
	#1
}{def}
\makeatother

\newtcbtheorem[number within=chapter]{wconc}{Wrong Concept}{
	breakable,
	enhanced,
	colback=white,
	colframe=myr,
	arc=0pt,
	outer arc=0pt,
	fonttitle=\bfseries\sffamily\large,
	colbacktitle=myr,
	attach boxed title to top left={},
	boxed title style={
			enhanced,
			skin=enhancedfirst jigsaw,
			arc=3pt,
			bottom=0pt,
			interior style={fill=myr}
		},
	#1
}{def}



%================================
% NOTE BOX
%================================

\usetikzlibrary{arrows,calc,shadows.blur}
\tcbuselibrary{skins}
\newtcolorbox{note}[1][]{%
	enhanced jigsaw,
	colback=gray!20!white,%
	colframe=gray!80!black,
	size=small,
	boxrule=1pt,
	title=\textbf{Note},
	halign title=flush center,
	coltitle=black,
	breakable,
	drop shadow=black!50!white,
	attach boxed title to top left={xshift=1cm,yshift=-\tcboxedtitleheight/2,yshifttext=-\tcboxedtitleheight/2},
	minipage boxed title=1.5cm,
	boxed title style={%
			colback=white,
			size=fbox,
			boxrule=1pt,
			boxsep=2pt,
			underlay={%
					\coordinate (dotA) at ($(interior.west) + (-0.5pt,0)$);
					\coordinate (dotB) at ($(interior.east) + (0.5pt,0)$);
					\begin{scope}
						\clip (interior.north west) rectangle ([xshift=3ex]interior.east);
						\filldraw [white, blur shadow={shadow opacity=60, shadow yshift=-.75ex}, rounded corners=2pt] (interior.north west) rectangle (interior.south east);
					\end{scope}
					\begin{scope}[gray!80!black]
						\fill (dotA) circle (2pt);
						\fill (dotB) circle (2pt);
					\end{scope}
				},
		},
	#1,
}

%================================
% Clarification box
%================================

\newtcbtheorem[number within=section]{Clarification}{Clarification}
{%
	colback = myclarificationbg,
	breakable,
	colframe = myclarificationfr,
	coltitle = myexampleti,
	boxrule = 1pt,
	sharp corners,
	detach title,
	before upper=\tcbtitle\par\smallskip,
	fonttitle = \bfseries,
	description font = \mdseries,
	separator sign none,
	description delimiters parenthesis
}
{clarif}

\newtcbtheorem[number within=chapter]{clarification}{Clarification}
{%
	colback = myclarificationbg,
	% breakable,
	colframe = myclarificationfr,
	coltitle = myexampleti,
	boxrule = 1pt,
	sharp corners,
	detach title,
	before upper=\tcbtitle\par\smallskip,
	fonttitle = \bfseries,
	description font = \mdseries,
	separator sign none,
	description delimiters
}
{clarif}

%%%%%%%%%%%%%%%%%%%%%%%%%%%%%%
% SELF MADE COMMANDS
%%%%%%%%%%%%%%%%%%%%%%%%%%%%%%

\newcommand{\nt}[1]{\begin{note}#1\end{note}}

\newcommand{\mcthmBreakable}[2]{\begin{Theorem*}{#1}{}#2\end{Theorem*}}
\newcommand{\mcthm}[2]{\begin{theorem*}{#1}{}#2\end{theorem*}}
\newcommand{\mccorollarBreakable}[2]{\begin{Corollary*}{#1}{}#2\end{Corollary*}}
\newcommand{\mccorollar}[2]{\begin{corollary*}{#1}{}#2\end{corollary*}}
\newcommand{\mcpropBreakable}[2]{\begin{Prop*}{#1}{}#2\end{Prop*}}
\newcommand{\mcprop}[2]{\begin{prop*}{#1}{}#2\end{prop*}}
\newcommand{\mcdfnBreakable}[2]{\begin{Definition*}[colbacktitle=red!75!black]{#1}{}#2\end{Definition*}}
\newcommand{\mcdfn}[2]{\begin{definition*}[colbacktitle=red!75!black]{#1}{}#2\end{definition*}}
\newcommand{\mcexBreakable}[2]{\begin{Example*}{#1}{}#2\end{Example*}}
\newcommand{\mcex}[2]{\begin{example*}{#1}{}#2\end{example*}}
\newcommand{\mcprf}[1]{\begin{mcproof}\[#1\]\end{mcproof}}
\newcommand{\mcproofpure}[1]{\begin{mcproof}#1\end{mcproof}}
\newcommand{\mcclmBreakable}[3]{\begin{Claim*}{#1}{#2}#3\end{Claim*}}
\newcommand{\mcclm}[3]{\begin{claim*}{#1}{#2}#3\end{claim*}}
\newcommand{\mcclarfBreakable}[2]{\begin{Clarification*}{#1}{}#2\end{Clarification*}}
\newcommand{\mcclarf}[2]{\begin{clarification*}{#1}{}#2\end{clarification*}}
\newcommand{\mclemmaBreakable}[2]{\begin{Lemma*}{#1}{}#2\end{Lemma*}}
\newcommand{\mclemma}[2]{\begin{lemma*}{#1}{}#2\end{lemma*}}

\newenvironment{myclaim}[1][\claimname]{\proof[\bfseries #1: ]}{}

\newcommand*\circled[1]{\tikz[baseline=(char.base)]{
		\node[shape=circle,draw,inner sep=1pt] (char) {#1};}}
\newcommand\getcurrentref[1]{%
	\ifnumequal{\value{#1}}{0}
	{??}
	{\the\value{#1}}%
}
\newcommand{\getCurrentSectionNumber}{\getcurrentref{section}}
\newenvironment{mcproof}[1][\proofname]{%
	\proof[\bfseries #1: ]%
}{\endproof}


\newcounter{mylabelcounter}

\makeatletter
\newcommand{\setword}[2]{%
	\phantomsection
	#1\def\@currentlabel{\unexpanded{#1}}\label{#2}%
}
\makeatother


\tikzset{
	symbol/.style={
			draw=none,
			every to/.append style={
					edge node={node [sloped, allow upside down, auto=false]{$#1$}}}
		}
}


% deliminators
% \DeclarePairedDelimiter{\abs}{\lvert}{\rvert}  % declared in <physics>
% \DeclarePairedDelimiter{\norm}{\lVert}{\rVert} % declared in <physics>

\DeclarePairedDelimiter{\ceil}{\lceil}{\rceil}
\DeclarePairedDelimiter{\floor}{\lfloor}{\rfloor}
\DeclarePairedDelimiter{\round}{\lfloor}{\rceil}

\newsavebox\diffdbox
\newcommand{\slantedromand}{{\mathpalette\makesl{d}}}
\newcommand{\makesl}[2]{%
\begingroup
\sbox{\diffdbox}{$\mathsurround=0pt#1\mathrm{#2}$}%
\pdfsave
\pdfsetmatrix{1 0 0.2 1}%
\rlap{\usebox{\diffdbox}}%
\pdfrestore
\hskip\wd\diffdbox
\endgroup
}
\newcommand{\mcdd}[1][]{\ensuremath{\mathop{}\!\ifstrempty{#1}{%
\slantedromand\@ifnextchar^{\hspace{0.2ex}}{\hspace{0.1ex}}}%
{\slantedromand\hspace{0.2ex}^{#1}}}}
\ProvideDocumentCommand\dv{o m g}{%
  \ensuremath{%
    \IfValueTF{#3}{%
      \IfNoValueTF{#1}{%
        \frac{\mcdd #2}{\mcdd #3}%
      }{%
        \frac{\mcdd^{#1} #2}{\mcdd #3^{#1}}%
      }%
    }{%
      \IfNoValueTF{#1}{%
        \frac{\mcdd}{\mcdd #2}%
      }{%
        \frac{\mcdd^{#1}}{\mcdd #2^{#1}}%
      }%
    }%
  }%
}
\providecommand*{\pdv}[3][]{\frac{\partial^{#1}#2}{\partial#3^{#1}}}

% Since the amsthm package isn't loaded

% I prefer the slanted \leq
\let\oldleq\leq % save them in case they're every wanted
\let\oldgeq\geq
\renewcommand{\leq}{\leqslant}
\renewcommand{\geq}{\geqslant}

% % redefine matrix env to allow for alignment, use r as default
% \renewcommand*\env@matrix[1][r]{\hskip -\arraycolsep
%     \let\@ifnextchar\new@ifnextchar
%     \array{*\c@MaxMatrixCols #1}}


%\usepackage{framed}
%\usepackage{titletoc}
%\usepackage{etoolbox}
%\usepackage{lmodern}


%\patchcmd{\tableofcontents}{\contentsname}{\sffamily\contentsname}{}{}

%\renewenvironment{leftbar}
%{\def\FrameCommand{\hspace{6em}%
%		{\color{myyellow}\vrule width 2pt depth 6pt}\hspace{1em}}%
%	\MakeFramed{\parshape 1 0cm \dimexpr\textwidth-6em\relax\FrameRestore}\vskip2pt%
%}
%{\endMakeFramed}

%\titlecontents{chapter}
%[0em]{\vspace*{2\baselineskip}}
%{\parbox{4.5em}{%
%		\hfill\Huge\sffamily\bfseries\color{myred}\thecontentspage}%
%	\vspace*{-2.3\baselineskip}\leftbar\textsc{\small\chaptername~\thecontentslabel}\\\sffamily}
%{}{\endleftbar}
%\titlecontents{section}
%[8.4em]
%{\sffamily\contentslabel{3em}}{}{}
%{\hspace{0.5em}\nobreak\itshape\color{myred}\contentspage}
%\titlecontents{subsection}
%[8.4em]
%{\sffamily\contentslabel{3em}}{}{}  
%{\hspace{0.5em}\nobreak\itshape\color{myred}\contentspage}


%%%%%%%%%%%%%%%%%%%%%%%%%%%%%%%%%%%%%%%%%%%
% TABLE OF CONTENTS
%%%%%%%%%%%%%%%%%%%%%%%%%%%%%%%%%%%%%%%%%%%

\usepackage{tikz}
\definecolor{doc}{RGB}{0,60,110}
\usepackage{titletoc}

% \contentsmargin{0cm}
% \titlecontents{chapter}[3.7pc]
% {\addvspace{30pt}%
% 	\begin{tikzpicture}[remember picture, overlay]%
% 		\draw[fill=doc!60,draw=doc!60] (-7,-.1) rectangle (-0.9,.5);%
% 		\pgftext[left,x=-3.5cm,y=0.2cm]{\color{white}\Large\sc\bfseries Chapter\ \thecontentslabel};%
% 	\end{tikzpicture}\color{doc!60}\large\sc\bfseries}%
% {}
% {}
% {\;\titlerule\;\large\sc\bfseries Page \thecontentspage
% 	\begin{tikzpicture}[remember picture, overlay]
% 		\draw[fill=doc!60,draw=doc!60] (2pt,0) rectangle (4,0.1pt);
% 	\end{tikzpicture}}%
% \titlecontents{section}[3.7pc]
% {\addvspace{2pt}}
% {\contentslabel[\thecontentslabel]{2pc}}
% {}
% {\hfill\small \thecontentspage}
% []
% \titlecontents*{subsection}[3.7pc]
% {\addvspace{-1pt}\small}
% {}
% {}
% {\ --- \small\thecontentspage}
% [ \textbullet\ ][]

% \makeatletter
% \renewcommand{\tableofcontents}{%
% 	\chapter*{%
% 	  \vspace*{-20\p@}%
% 	  \begin{tikzpicture}[remember picture, overlay]%
% 		  \pgftext[right,x=15cm,y=0.2cm]{\color{doc!60}\Huge\sc\bfseries \contentsname};%
% 		  \draw[fill=doc!60,draw=doc!60] (13,-.75) rectangle (20,1);%
% 		  \clip (13,-.75) rectangle (20,1);
% 		  \pgftext[right,x=15cm,y=0.2cm]{\color{white}\Huge\sc\bfseries \contentsname};%
% 	  \end{tikzpicture}}%
% 	\@starttoc{toc}}
% \makeatother


%From M275 "Topology" at SJSU
\newcommand{\id}{\mathrm{id}}
\newcommand{\taking}[1]{\xrightarrow{#1}}
\newcommand{\inv}{^{-1}}

%From M170 "Introduction to Graph Theory" at SJSU
\DeclareMathOperator{\diam}{diam}
\DeclareMathOperator{\ord}{ord}
\newcommand{\defeq}{\overset{\mathrm{def}}{=}}

%From the USAMO .tex files
\newcommand{\ts}{\textsuperscript}
\newcommand{\dg}{^\circ}
\newcommand{\ii}{\item}

% % From Math 55 and Math 145 at Harvard
% \newenvironment{subproof}[1][Proof]{%
% \begin{proof}[#1] \renewcommand{\qedsymbol}{$\blacksquare$}}%
% {\end{proof}}

\newcommand{\liff}{\leftrightarrow}
\newcommand{\lthen}{\rightarrow}
\newcommand{\opname}{\operatorname}
\newcommand{\surjto}{\twoheadrightarrow}
\newcommand{\injto}{\hookrightarrow}
\newcommand{\On}{\mathrm{On}}        % ordinals
\DeclareMathOperator{\img}{im}       % Image
\DeclareMathOperator{\Img}{Im}       % Image
\DeclareMathOperator{\coker}{coker}  % Cokernel
\DeclareMathOperator{\Coker}{Coker}  % Cokernel
\DeclareMathOperator{\Ker}{Ker}      % Kernel
% \DeclareMathOperator{\rank}{rank}  % rank       % declared in <physics>
\DeclareMathOperator{\Spec}{Spec}    % spectrum
% \DeclareMathOperator{\Tr}{Tr}      % trace      % declared in <physics>
\DeclareMathOperator{\pr}{pr}        % projection
\DeclareMathOperator{\ext}{ext}      % extension
\DeclareMathOperator{\pred}{pred}    % predecessor
\DeclareMathOperator{\dom}{dom}      % domain
\DeclareMathOperator{\ran}{ran}      % range
\DeclareMathOperator{\Hom}{Hom}      % homomorphism
\DeclareMathOperator{\Mor}{Mor}      % morphisms
\DeclareMathOperator{\End}{End}      % endomorphism

\DeclareMathOperator{\lowlim}{\underline{lim}} % lower limit
\DeclareMathOperator{\uplim}{\overline{lim}}   % upper limit

% combinations
\newcommand{\Cmb}[2]{ C^{#2}_{#1} }
\newcommand{\NumSeq}[1]{\{ #1 \}}
\newcommand{\SmO}[1][x]{\overline{o}(#1)}

%probabilities
\newcommand{\Proba}[1]{\mathds{P}\left( #1 \right)}
\newcommand{\Probc}[2]{\mathds{P}\left( #1 ~\vert~ #2\right)}

%expected values
\newcommand{\E}[1]{\mathds{E} \left( #1 \right) }
\newcommand{\Ec}[2]{\mathds{E} \left[ #1 ~\vert ~ #2\right] }

% - others
\DeclareMathOperator{\Lap}{\mathcal{L}}
\DeclareMathOperator{\Var}{var} % varience
\DeclareMathOperator{\cov}{cov} % covarience

\newcommand{\veps}{\varepsilon}
\newcommand{\wt}{\widetilde}
\newcommand{\wh}{\widehat}
\newcommand{\vocab}[1]{\textbf{\color{blue} #1}}
\providecommand{\half}{\frac{1}{2}}
\newcommand{\dang}{\measuredangle} %% Directed angle
\newcommand{\ray}[1]{\overrightarrow{#1}}
\newcommand{\seg}[1]{\overline{#1}}
\newcommand{\arc}[1]{\wideparen{#1}}
\DeclareMathOperator{\cis}{cis}
\DeclareMathOperator*{\lcm}{lcm}

\DeclareMathOperator*{\argmin}{arg min}
\DeclareMathOperator*{\argmax}{arg max}

% Contradiction symbol (like \bot)
\newcommand{\Contradiction}{\ensuremath{\circled{$\mathbb{W}$}}}

\newcommand{\cycsum}{\sum_{\mathrm{cyc}}}
\newcommand{\symsum}{\sum_{\mathrm{sym}}}
\newcommand{\cycprod}{\prod_{\mathrm{cyc}}}
\newcommand{\symprod}{\prod_{\mathrm{sym}}}
\newcommand{\Qed}{\begin{flushright}\qed\end{flushright}}
\newcommand{\parinn}{\setlength{\parindent}{1cm}}
\newcommand{\parinf}{\setlength{\parindent}{0cm}}
% \newcommand{\norm}{\|\cdot\|}                         % declared in <physics>
\newcommand{\inorm}{\norm_{\infty}}
\newcommand{\opensets}{\{V_{\alpha}\}_{\alpha\in I}}
\newcommand{\oset}{V_{\alpha}}
\newcommand{\opset}[1]{V_{\alpha_{#1}}}
\newcommand{\lub}{\text{lub}}
\newcommand{\del}[2]{\frac{\partial #1}{\partial #2}}
\newcommand{\Del}[3]{\frac{\partial^{#1} #2}{\partial^{#1} #3}}
\newcommand{\deld}[2]{\dfrac{\partial #1}{\partial #2}}
\newcommand{\Deld}[3]{\dfrac{\partial^{#1} #2}{\partial^{#1} #3}}
\newcommand{\lm}{\lambda}
\newcommand{\uin}{\mathbin{\rotatebox[origin=c]{90}{$\in$}}}
\newcommand{\usubset}{\mathbin{\rotatebox[origin=c]{90}{$\subset$}}}
\newcommand{\lt}{\left}
\newcommand{\rt}{\right}
\newcommand{\bs}[1]{\boldsymbol{#1}}
\newcommand{\exs}{\exists}
\newcommand{\st}{\strut}
\newcommand{\dps}[1]{\displaystyle{#1}}

\newcommand{\sol}{\setlength{\parindent}{0cm}\textbf{\textit{Solution:}}\setlength{\parindent}{1cm} }
\newcommand{\solve}[1]{\setlength{\parindent}{0cm}\textbf{\textit{Solution: }}\setlength{\parindent}{1cm}#1 \Qed}


\DeclareMathOperator{\Hash}{Hash}
\DeclareMathOperator{\hash}{hash}

\begin{document}

\section*{Theory}

\mcclm{Утверждение 1}{}{
    В данной хэш функции используется знаковая арифметика, тип \textbf{long long}

    Считая, что используется беззнаковая арифметика, то есть тип \textbf{unsigned long long},
    результат работы хэш функции не изменится (то есть по-битово значения будут совпадать)
\mcprf{
\begin{split}
    & \text{В хэш функции используются только операции умножения и сложения } \\
    & \text{Тогда результат их работы не отличается для знаковых и беззнаковых типов } \\
    & \text{ввиду реализации сложения и умножения в АЛУ (см. инструкции \textbf{add} и \textbf{mul}) } \\
\end{split}
}
}

\mcclarf{Уточнение}{
    Далее в рассуждениях используются следующие обозначения:

\begin{tabular}{rl}
    $\bullet$ & $ \Sigma = \{ 'a', 'b', ..., 'z' \} $ - символы таблицы ASCII с кодами от 97 до 122 \\
    $\bullet$ & $ \sigma = \{ 1, 2, ..., 26 \} $ \\
              & При дальнейшем описании и построении искомого алгоритма будем считать, \\
              & что используются только символы таблицы ASCII из множества $ \Sigma $ \\
    $\bullet$ & $ \Sigma^* $ - замыкание Клини, т.е. множество всех слов конечной длины, \\
              & составленных из символов множества $\Sigma$, в том числе пустое слово $\veps$, $|\veps| = 0 $ \\
    $\bullet$ & $ \Sigma^+ $ - плюс Клини, т.е. множество всех слов конечной длины, \\
              & составленных из символов множества $\Sigma$, без пустого слова $\veps$ \\
              & $ \Sigma^+ = \Sigma^* \setminus \{ \veps \} $ \\
    $\bullet$ & $ p $ - данное простое число, $ p = 31 $ \\
    $\bullet$ & $ \Hash(s) $ - математическое значение полиномиальной хэш функции строки $s \in \Sigma^*$ \\
              & без операция взятия остатка по модулю $2^{64}$ \\
    $\bullet$ & $ \hash(s) $ - значение хэш функции строки $s \in \Sigma^*$ по модулю $2^{64}$, т.е. \\
              & $ \forall s \in \Sigma^*: \hash(s) = \Hash(S) \bmod 2^{64} $ \\
              & По утверждению 1 можно считать, что данная полиномиальная хэш функция - $\hash$ \\
    $\bullet$ & $ \overline{s_0 s_1 ... s_{n-1}} $ - строка, состоящая из последовательности строк $\{ s_i \}_{i = 0}^{n - 1}$, \\
              & последовательно сконкатенированных в одну строку \\
              & (мы не определяем это формально, потому что \sout{автору лень} тогда мы утонем в формализме \\
              & при доказательстве задачи, которая решается за 15 минут\sout{, а ещё автору лень}) \\
\end{tabular}
}

\mclemma{Лемма 1}{
    При подсчёте хэш функции $\Hash$ от строки $s \in \Sigma^*$ получается число в системе счисления по основанию $p$,
    где цифры числа $ \Hash(s) $ в с.с. по основанию $p$ соответствуют символам строки $s$ и принадлежат множеству $ \sigma $ 
\mcprf{
\begin{split}
    & \text{Следует из того, что при подсчёте хэш функции } \forall i \in \{ 0, 1, ..., |s| - 1 \}: s[i] - 'a' + 1 \in \sigma \subset \{ 0, 1, ..., p - 1\} \\
    & \text{И при этом $s[i] - 'a' + 1$ умножается на } p^i: \Hash(s) = \sum_{i = 0}^{|s| - 1} (s[i] - 'a' + 1) \cdot p^i  \\
\end{split}
}
}

\pagebreak

\mclemma{Лемма 2}{
    Пусть существует такая строка $s_0 \in \Sigma^+ $ длины $|s_0|$, что $ \hash(s_0) = 0 $

    Тогда $ \forall n \in \Nset[]: \hash(\overline{\underset{n \text{ раз}}{s_0 s_0 ... s_0}}) = 0 $

\mcprf{
\begin{split}
    & \text{1. По построению хэш функции $\Hash$:} \\
    & \Hash(s_0) = \sum_{i = 0}^{|s_0| - 1} (s_0[i] - 'a' + 1) \cdot p^i \\
    & \text{2. Тогда: } \\
    & \Hash(\overline{s_0 s_0 ... s_0}) 
        = \sum_{i = 0}^{n \cdot |s_0| - 1} (\overline{s_0 s_0 ... s_0}[i] + 'a' - 1) \cdot p^i = \\
    &   = \sum_{j = 0}^{n - 1} \sum_{i = |s_0| \cdot j}^{|s_0| \cdot (j + 1) - 1} (\overline{s_0 s_0 ... s_0}[i] + 'a' - 1) \cdot p^i = \\
    &   = \sum_{j = 0}^{n - 1} p^{|s_0| \cdot j} \sum_{i = |s_0| \cdot j}^{|s_0| \cdot (j + 1) - 1} (\overline{s_0 s_0 ... s_0}[i] + 'a' - 1) \cdot p^{i - |s_0| \cdot j} = \\
    &   = \sum_{j = 0}^{n - 1} p^{|s_0| \cdot j} \sum_{i = 0}^{|s_0| - 1} (\overline{s_0 s_0 ... s_0}[i + |s_0| \cdot j] + 'a' - 1) \cdot p^{i} = \\
    &   = \sum_{j = 0}^{n - 1} p^{|s_0| \cdot j} \sum_{i = 0}^{|s_0| - 1} (s_0[i] + 'a' - 1) \cdot p^{i} 
        = \sum_{j = 0}^{n - 1} p^{|s_0| \cdot j} \cdot \Hash(s_0)
        = \Hash(s_0) \cdot \sum_{j = 0}^{n - 1} p^{|s_0| \cdot j} \\
    & \hash(s_0) = 0 
        \implies \Hash(s_0) \bmod 2^{64} = 0 
        \implies \hash(\overline{s_0 s_0 ... s_0}) = \Hash(\overline{s_0 s_0 ... s_0}) \bmod 2^{64} = \\
    & = \left(\Hash(s_0) \cdot \sum_{j = 0}^{n - 1} p^{|s_0| \cdot j}\right) \bmod 2^{64} = 0 \\
\end{split}
}
}

\pagebreak

\mclemma{Лемма 3}{
    Пусть существует такая строка $s_0 \in \Sigma^+$ длины $|s_0|$, что $ \hash(s_0) = 0 $

    Тогда $ \forall s \in \Sigma^* \, \forall n \in \Nset[]: \hash(\overline{s \underset{n \text{ раз}}{s_0 s_0 ... s_0}}) = \hash(s) $
\mcprf{
\begin{split}
    & \text{1. По построению хэш функции: } \\
    & \Hash(\overline{s s_0 s_0 ... s_0}) 
        = \sum_{i = 0}^{|s| + n \cdot |s_0| - 1} (\overline{s s_0 s_0 ... s_0}[i] - 'a' + 1) \cdot p^i = \\
    &   = \sum_{i = 0}^{|s| - 1} (\overline{s s_0 s_0 ... s_0}[i] + 'a' - 1) \cdot p^i 
        + \sum_{i = |s|}^{|s| + n \cdot |s_0| - 1} (\overline{s s_0 s_0 ... s_0}[i] + 'a' - 1) \cdot p^i  \\
    &   = \sum_{i = 0}^{|s| - 1} (s[i] + 'a' - 1) \cdot p^i 
        + p^{|s|} \cdot \sum_{i = |s|}^{|s| + n \cdot |s_0| - 1} (\overline{s s_0 s_0 ... s_0}[i] + 'a' - 1) \cdot p^{i - |s|} = \\
    &   = \Hash(s) + p^{|s|} \cdot \sum_{i = |s|}^{|s| + n \cdot |s_0| - 1} (\overline{s s_0 s_0 ... s_0}[i] + 'a' - 1) \cdot p^{i - |s|} = \\
    &   = \Hash(s) + p^{|s|} \cdot \sum_{i = 0}^{n \cdot |s_0| - 1} (\overline{s s_0 s_0 ... s_0}[i + |s|] + 'a' - 1) \cdot p^i = \\
    &   = \Hash(s) + p^{|s|} \cdot \sum_{i = 0}^{n \cdot |s_0| - 1} (\overline{s_0 s_0 ... s_0}[i] + 'a' - 1) \cdot p^i = \\
    &   = \Hash(s) + p^{|s|} \cdot \Hash(\overline{s_0 s_0 ... s_0}) \\
    & \text{2. по лемме 2 $\hash(\overline{s_0 s_0 ... s_0}) = 0 $, тогда:} \\
    & \hash(\overline{s s_0 s_0 ... s_0}) 
        = \Hash(\overline{s s_0 s_0 ... s_0}) \bmod 2^{64}
        = \Hash(s) \bmod 2^{64} + \left(p^{|s|} \cdot \Hash(\overline{s_0 s_0 ... s_0})\right) \bmod 2^{64} = \\
    &   = \hash(s) + \left(p^{|s|} \cdot \hash(\overline{s_0 s_0 ... s_0})\right) \bmod 2^{64} 
        = \hash(s) \\
\end{split}
}
}

\mccorollar{Следствие}{
    Если существует хотя бы одна строка $s_0 \in \Sigma^*$, такая что $ |s_0| \ge 1 \wedge \hash(s_0) = 0$, то
    $\forall s \in \Sigma^*$ можно получить хотя бы счётно бесконечно много попарно различных строк, таких
    что их хэш равен $\hash(s)$, причём алгоритм построения искомой последовательности строк следует из леммы 3:
    члены последовательности получаются конкатенацией строки $s$ с $n$ строками $s_0$.
}

\mcclarf{Уточнение об имплементации алгоритма}{
    Если длина выбранной строки $s_0$ (для которой $\hash(s_0) = 0$) равна $|s_0|$ и нужно сгенерировать $N$ искомых 
    строк с хэшом, равным данной строке $s$ длины $|s|$, то асимптотически точная граница времени работы приведённого 
    ниже алгоритма равна $ \Theta\left(|s| + N |s_0|\right)$

    В приведённом ниже алгоритме строка $s_0$ (можно генерировать разные в зависимости от шаблонного параметра) 
    имеет длину не более $\textit{kMaxZeroRemStringSize} = 20$ и генерируется на этапе компиляции (если версия языка $\ge$ C++23).
}

\pagebreak

\section*{Practice}

Алгоритм реализован в файле \textit{collisions\_gen.hpp}

Публичный интерфейс для генерации строк находится в пространстве имён \text{collisions\_gen}

Реализация находится во вложенном пространстве имён collisions\_gen::impl

Публичный интерфейс:

\begin{tabular}{rl}
    $\bullet$ & \textit{polynomial\_hash} - данная хэш функция \\
    $\bullet$ & \textit{polynomial\_hash\_safe} - данная хэш функция, использующая беззнаковую арифметику \\
    $\bullet$ & \textit{generate\_strings\_with\_same\_hash} - функция для генерации \\
              & необходимого количества строк, хэш которых равен хэшу данной строки \\
              & Кроме 2 входных аргументов имеет шаблонный параметра Seed, который \\
              & может изменить генерирующиеся строки. По умолчанию равен \textit{impl::kDefaultStartN = 1} \\
              & следующее по возрастанию значение, меняющее выходные данные $ = 13$ \\
\end{tabular}

\begin{cppcode}

namespace collisions_gen {

/// @brief Default number of strings generated by @fn generate_strings_with_same_hash
inline constexpr uint32_t kDefaultSize = 2000;

/// @brief Polynomial hash function.
///        This function causes UB and hence not marked constexpr.
/// @param s
/// @return polynomial hash of the @a `s`
long long polynomial_hash(std::string_view s) noexcept;

/// @brief Polynomial hash function.
/// @param s
/// @return polynomial hash of the @a `s`
constexpr uint64_t polynomial_hash_safe(std::string_view s) noexcept;

/// @brief Functions that generates @a `size` strings with the same
///        hash as @a `str` has. Hash functions is @fn polynomial_hash.
/// @tparam Seed optional argument that may change generated set of string.
/// @param str Initial string.
///            If empty, and Seed = @ref impl::kDefaultStartN,
///            then @a `size` strings with hash = 0 will be generated.
/// @param size Number of strings that should be generated.
/// @return vector of @a `size` pairwise different strings,
///         each one has the same hash as @a `str` has.
template <uint32_t Seed = impl::kDefaultStartN>
std::vector<std::string> generate_strings_with_same_hash(std::string_view str = "", 
uint32_t size = kDefaultSize);

}  // namespace collisions_gen

\end{cppcode}


Пространство имён impl:

(некоторые детали реализации, например, поддержка более старых версий языка, убраны в коде ниже, полный код есть в файле collisions\_gen.hpp)

\begin{tabular}{rl}
    $\bullet$ & \textit{find\_zero\_rem\_num} - функция нахождения такого n, что все цифры числа $n \cdot 2^{64}$  \\   
              & в с.с. по основанию p принадлежат $\sigma$ \\
    $\bullet$ & \textit{ZeroHashStringBuffer} - шаблонный класс, который по данному числу $n$ переводит число $n \cdot 2^{64}$ \\   
              & в с.с. по основанию p и сохраняет в виде строки \\
    $\bullet$ & \textit{zero\_hash\_string} - шаблонная константа типа std::string\_view, такая, что её хэш равен 0 \\
\end{tabular}

\begin{cppcode}

namespace impl {

using uint128_t = __uint128_t;

inline constexpr uint32_t kPrime         = 31;
inline constexpr uint32_t kDefaultStartN = 1;
/// @brief @f[ \lceil log_{kPrime}( 2^{32} \cdot 2^{64} ) \rceil @f]
inline constexpr uint32_t kMaxZeroRemStringSize = 20;

/// @brief f: { 0, ..., 2^32 - 1 } -> { 0, 1 }
/// @param n
/// @return n |-> n \in [1; 26]
constexpr bool between_1_and_26(uint32_t n) noexcept;

/// @brief Checks whether all digits of n
///        in base @ref kPrime \in [1; 26]
/// @param n
/// @return
constexpr bool check_n_in_base_p(uint128_t n) noexcept;

/// @brief Finds integer n, such that all digits
///        of n * 2^64 in base @ref kPrime \in [1; 26]
/// @tparam StartN
/// @return n
template <uint32_t StartN = kDefaultStartN>
constexpr uint32_t find_zero_rem_num() noexcept;

template <uint32_t N>
struct ZeroHashStringBuffer {
    consteval ZeroHashStringBuffer() noexcept;
    consteval std::string_view as_string_view() const noexcept;
};

template <uint32_t StartN = kDefaultStartN>
inline constexpr ZeroHashStringBuffer<find_zero_rem_num<StartN>()> zero_hash_string_buffer;

template <uint32_t StartN = kDefaultStartN>
inline constexpr std::string_view zero_hash_string 
    = zero_hash_string_buffer<StartN>.as_string_view();

}  // namespace impl

\end{cppcode}

Вызов генератора строк находится в файле \textit{main.cpp}

В качестве первого аргумента исполняемому файлу можно передать начальную строку, к которой будут дописываться строки $s_0$

В качестве второго аргумента исполняемому файлу можно передать количество строк, которое надо сгенерировать

(также можно не передавать никаких параметров, будут выбраны дефолтные: пустая строка и 2000 строк)

\begin{cppcode}

void write_to_file(const std::vector<std::string>& strs, std::string_view fname);

std::pair<const char*, uint32_t> parse_arguments(int argc, const char* const argv[]) noexcept;

int main(int argc, const char* const argv[]) {
    auto [initial_string, size] = parse_arguments(argc, argv);
    auto res = collisions_gen::generate_strings_with_same_hash(initial_string, size);
    write_to_file(res, "strings.txt");
    return 0;
}

\end{cppcode}

\pagebreak

\section*{Tested compilers \& options}

При компиляции компилятором g++ 13.2.0 из среды msys2 на Windows 10 22H2 использовались следующие флаги:

-std=c++20, -std=c++2a, -std=c++2b для версий языка C++20 и C++23, а также:

\begin{cppcode}
-D_GLIBCXX_DEBUG
-D_GLIBCXX_DEBUG_PEDANTIC
-D_FORTIFY_SOURCE=3
-fdiagnostics-color=always
-fstack-protector-all
-mshstk
-Wall
-Wextra
-Wfloat-equal
-Wlogical-op
-Wcast-qual
-Wpedantic
-Wshift-overflow=2
-Wduplicated-cond
-Wunused -Wconversion
-Wunsafe-loop-optimizations
-Wshadow
-Wnull-dereference
-Wundef 
-Wwrite-strings
-Wsign-conversion
-Warith-conversion
-Wmissing-noreturn
-Wunreachable-code
-Wcast-align
-Warray-bounds=2
\end{cppcode}

При компиляции компилятором clang++ 16.0.5 из среды msys2 на Windows 10 22H2 использовались следующие флаги:

-std=c++20, -std=c++2a, -std=c++2b для версий языка C++20 и C++23, а также:

\begin{cppcode}
-fcolor-diagnostics
-fansi-escape-codes
-fsanitize="address,undefined"
-fstack-protector-all
-D_FORTIFY_SOURCE=3
-Wp,-D_GLIBCXX_DEBUG
-mshstk
-O2
-Wall
-Wextra
-Wpedantic
-Wunused
-Wconversion
-Wshadow
-Wnull-dereference
-Wundef
-Wwrite-strings
-Wsign-conversion
-Wmissing-noreturn
-Wunreachable-code
-Wcast-align
-Warray-bounds
\end{cppcode}

При компиляции компилятором g++ 13.2.0 на Ununtu 22.10 использовались следующие флаги:

-std=c++20, -std=c++2a, -std=c++2b, -std=c++23 для версий языка C++20 и C++23, а также:

\begin{cppcode}
-D_GLIBCXX_DEBUG
-D_GLIBCXX_DEBUG_PEDANTIC
-D_FORTIFY_SOURCE=3
-fdiagnostics-color=always
-fstack-protector-all
-fsanitize="address,undefined,leak"
-mshstk
-Wall
-Wextra
-Wfloat-equal
-Wlogical-op
-Wcast-qual
-Wpedantic
-Wshift-overflow=2
-Wduplicated-cond
-Wunused -Wconversion
-Wunsafe-loop-optimizations
-Wshadow
-Wnull-dereference
-Wundef 
-Wwrite-strings
-Wsign-conversion
-Warith-conversion
-Wmissing-noreturn
-Wunreachable-code
-Wcast-align
-Warray-bounds=2
\end{cppcode}

\end{document}
