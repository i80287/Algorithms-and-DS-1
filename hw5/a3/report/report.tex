\documentclass{report}
% \includeonly{pars/9_continuous_functions_and_limits, parts/10_integration_of_functions}
% \documentclass[11pt,a4paper]{article}

%%%%%%%%%%%%%%%%%%%%%%%%%%%%%%%%%
% PACKAGE IMPORTS
%%%%%%%%%%%%%%%%%%%%%%%%%%%%%%%%%

\usepackage[utf8]{inputenc} % Сурс - семинары по алгебре Медведя Никиты Юрьевича
\usepackage[T2A]{fontenc}

\usepackage[tmargin=2cm,rmargin=1in,lmargin=1in,margin=0.85in,bmargin=2cm,footskip=.2in]{geometry}
\usepackage{amsmath,amsfonts,amsthm,amssymb,mathtools}
\usepackage[varbb]{newpxmath}
\usepackage{xfrac}
\usepackage[makeroom]{cancel}
\usepackage{mathtools}
\usepackage{bookmark}
\usepackage{enumitem}
\usepackage{hyperref,theoremref}
\hypersetup{
	pdftitle={Assignment},
	colorlinks=true, linkcolor=doc!90,
	bookmarksnumbered=true,
	bookmarksopen=true
}
\usepackage[most,many,breakable]{tcolorbox}
\usepackage{xcolor}% http://ctan.org/pkg/xcolor
%\usepackage{colortbl}% http://ctan.org/pkg/colortbl
\usepackage{multirow}% http://ctan.org/pkg/multirow
\usepackage{graphicx}% http://ctan.org/pkg/graphicx
\usepackage{varwidth}
\usepackage{varwidth}
\usepackage{etoolbox}
%\usepackage{authblk}
\usepackage{nameref}
\usepackage{multicol,array}
\usepackage{tikz-cd}
\usepackage[ruled,vlined,linesnumbered]{algorithm2e}
\usepackage{comment} % enables the use of multi-line comments (\ifx \fi) 
\usepackage{import}
\usepackage{xifthen}
\usepackage{pdfpages}
\usepackage{transparent}

\usepackage[english]{babel}
% ,russian
\usepackage{amsfonts,amssymb}
\usepackage{relsize}

\usepackage{systeme}

\usepackage{indentfirst} % Красная строка
\usepackage{fancyhdr}
\usepackage{wrapfig}
\usepackage{textcomp}

% \usepackage[unicode]{hyperref}

\newcommand\mycommfont[1]{\footnotesize\ttfamily\textcolor{blue}{#1}}
\SetCommentSty{mycommfont}
\newcommand{\incfig}[1]{%
    \def\svgwidth{\columnwidth}
    \import{./figures/}{#1.pdf_tex}
}

\usepackage{tikzsymbols}
\renewcommand\qedsymbol{$\Laughey$}


%\usepackage{import}
%\usepackage{xifthen}
%\usepackage{pdfpages}
%\usepackage{transparent}


%%%%%%%%%%%%%%%%%%%%%%%%%%%%%%
% SELF MADE COLORS
%%%%%%%%%%%%%%%%%%%%%%%%%%%%%%



\definecolor{myg}{RGB}{56, 140, 70}
\definecolor{myb}{RGB}{45, 111, 177}
\definecolor{myr}{RGB}{199, 68, 64}
\definecolor{mytheorembg}{HTML}{F2F2F9}
\definecolor{mytheoremfr}{HTML}{00007B}
\definecolor{mylenmabg}{HTML}{FFFAF8}
\definecolor{mylenmafr}{HTML}{983b0f}
\definecolor{mypropbg}{HTML}{f2fbfc}
\definecolor{mypropfr}{HTML}{191971}
\definecolor{myexamplebg}{HTML}{F2FBF8}
\definecolor{myexamplefr}{HTML}{88D6D1}
\definecolor{myexampleti}{HTML}{2A7F7F}
\definecolor{mydefinitbg}{HTML}{E5E5FF}
\definecolor{mydefinitfr}{HTML}{3F3FA3}
\definecolor{notesgreen}{RGB}{0,162,0}
\definecolor{myp}{RGB}{197, 92, 212}
\definecolor{mygr}{HTML}{2C3338}
\definecolor{myred}{RGB}{127,0,0}
\definecolor{myyellow}{RGB}{169,121,69}
\definecolor{myexercisebg}{HTML}{F2FBF8}
\definecolor{myexercisefg}{HTML}{88D6D1}

%%%%%%%%%%%%%%%%%%%%%%%%%%%%%%%%%%%%%%%%%%%
% TABLE OF CONTENTS
%%%%%%%%%%%%%%%%%%%%%%%%%%%%%%%%%%%%%%%%%%%

\usepackage{tikz}
\definecolor{doc}{RGB}{0,60,110}
\usepackage{titletoc}
\contentsmargin{0cm}
\titlecontents{chapter}[3.7pc]
{\addvspace{30pt}%
	\begin{tikzpicture}[remember picture, overlay]%
		\draw[fill=doc!60,draw=doc!60] (-7,-.1) rectangle (-0.9,.5);%
		\pgftext[left,x=-3.5cm,y=0.2cm]{\color{white}\Large\sc\bfseries Chapter\ \thecontentslabel};%
	\end{tikzpicture}\color{doc!60}\large\sc\bfseries}%
{}
{}
{\;\titlerule\;\large\sc\bfseries Page \thecontentspage
	\begin{tikzpicture}[remember picture, overlay]
		\draw[fill=doc!60,draw=doc!60] (2pt,0) rectangle (4,0.1pt);
	\end{tikzpicture}}%
\titlecontents{section}[3.7pc]
{\addvspace{2pt}}
{\contentslabel[\thecontentslabel]{2pc}}
{}
{\hfill\small \thecontentspage}
[]
\titlecontents*{subsection}[3.7pc]
{\addvspace{-1pt}\small}
{}
{}
{\ --- \small\thecontentspage}
[ \textbullet\ ][]

\makeatletter
\renewcommand{\tableofcontents}{%
	\chapter*{%
	  \vspace*{-20\p@}%
	  \begin{tikzpicture}[remember picture, overlay]%
		  \pgftext[right,x=15cm,y=0.2cm]{\color{doc!60}\Huge\sc\bfseries \contentsname};%
		  \draw[fill=doc!60,draw=doc!60] (13,-.75) rectangle (20,1);%
		  \clip (13,-.75) rectangle (20,1);
		  \pgftext[right,x=15cm,y=0.2cm]{\color{white}\Huge\sc\bfseries \contentsname};%
	  \end{tikzpicture}}%
	\@starttoc{toc}}
\makeatother


\input{macros}

\DeclareMathOperator{\Hash}{Hash}
\DeclareMathOperator{\hash}{hash}

\begin{document}

\section*{Theory}

\mcclm{Утверждение 1}{}{
    В данной хэш функции используется знаковая арифметика, тип \textbf{long long}

    Считая, что используется беззнаковая арифметика, то есть тип \textbf{unsigned long long},
    результат работы хэш функции не изменится (то есть по-битово значения будут совпадать)
\mcprf{
\begin{split}
    & \text{В хэш функции используются только операции умножения и сложения } \\
    & \text{Тогда результат их работы не отличается для знаковых и беззнаковых типов } \\
    & \text{ввиду реализации сложения и умножения в АЛУ (см. инструкции \textbf{add} и \textbf{mul}) } \\
\end{split}
}
}

\mcclarf{Уточнение}{
    Далее в рассуждениях используются следующие обозначения:

\begin{tabular}{rl}
    $\bullet$ & $ \Sigma = \{ 'a', 'b', ..., 'z' \} $ - символы таблицы ASCII с кодами от 97 до 122 \\
    $\bullet$ & $ \sigma = \{ 1, 2, ..., 26 \} $ \\
              & При дальнейшем описании и построении искомого алгоритма будем считать, \\
              & что используются только символы таблицы ASCII из множества $ \Sigma $ \\
    $\bullet$ & $ \Sigma^* $ - замыкание Клини, т.е. множество всех слов конечной длины, \\
              & составленных из символов множества $\Sigma$, в том числе пустое слово $\veps$, $|\veps| = 0 $ \\
    $\bullet$ & $ \Sigma^+ $ - плюс Клини, т.е. множество всех слов конечной длины, \\
              & составленных из символов множества $\Sigma$, без пустого слова $\veps$ \\
              & $ \Sigma^+ = \Sigma^* \setminus \{ \veps \} $ \\
    $\bullet$ & $ p $ - данное простое число, $ p = 31 $ \\
    $\bullet$ & $ \Hash(s) $ - математическое значение полиномиальной хэш функции строки $s \in \Sigma^*$ \\
              & без операция взятия остатка по модулю $2^{64}$ \\
    $\bullet$ & $ \hash(s) $ - значение хэш функции строки $s \in \Sigma^*$ по модулю $2^{64}$, т.е. \\
              & $ \forall s \in \Sigma^*: \hash(s) = \Hash(S) \bmod 2^{64} $ \\
              & По утверждению 1 можно считать, что данная полиномиальная хэш функция - $\hash$ \\
    $\bullet$ & $ \overline{s_0 s_1 ... s_{n-1}} $ - строка, состоящая из последовательности строк $\{ s_i \}_{i = 0}^{n - 1}$, \\
              & последовательно сконкатенированных в одну строку \\
              & (мы не определяем это формально, потому что \sout{автору лень} тогда мы утонем в формализме \\
              & при доказательстве задачи, которая решается за 15 минут\sout{, а ещё автору лень}) \\
\end{tabular}
}

\mclemma{Лемма 1}{
    При подсчёте хэш функции $\Hash$ от строки $s \in \Sigma^*$ получается число в системе счисления по основанию $p$,
    где цифры числа $ \Hash(s) $ в с.с. по основанию $p$ соответствуют символам строки $s$ и принадлежат множеству $ \sigma $ 
\mcprf{
\begin{split}
    & \text{Следует из того, что при подсчёте хэш функции } \forall i \in \{ 0, 1, ..., |s| - 1 \}: s[i] - 'a' + 1 \in \sigma \subset \{ 0, 1, ..., p - 1\} \\
    & \text{И при этом $s[i] - 'a' + 1$ умножается на } p^i: \Hash(s) = \sum_{i = 0}^{|s| - 1} (s[i] - 'a' + 1) \cdot p^i  \\
\end{split}
}
}

\pagebreak

\mclemma{Лемма 2}{
    Пусть существует такая строка $s_0 \in \Sigma^+ $ длины $|s_0|$, что $ \hash(s_0) = 0 $

    Тогда $ \forall n \in \Nset[]: \hash(\overline{\underset{n \text{ раз}}{s_0 s_0 ... s_0}}) = 0 $

\mcprf{
\begin{split}
    & \text{1. По построению хэш функции $\Hash$:} \\
    & \Hash(s_0) = \sum_{i = 0}^{|s_0| - 1} (s_0[i] - 'a' + 1) \cdot p^i \\
    & \text{2. Тогда: } \\
    & \Hash(\overline{s_0 s_0 ... s_0}) 
        = \sum_{i = 0}^{n \cdot |s_0| - 1} (\overline{s_0 s_0 ... s_0}[i] + 'a' - 1) \cdot p^i = \\
    &   = \sum_{j = 0}^{n - 1} \sum_{i = |s_0| \cdot j}^{|s_0| \cdot (j + 1) - 1} (\overline{s_0 s_0 ... s_0}[i] + 'a' - 1) \cdot p^i = \\
    &   = \sum_{j = 0}^{n - 1} p^{|s_0| \cdot j} \sum_{i = |s_0| \cdot j}^{|s_0| \cdot (j + 1) - 1} (\overline{s_0 s_0 ... s_0}[i] + 'a' - 1) \cdot p^{i - |s_0| \cdot j} = \\
    &   = \sum_{j = 0}^{n - 1} p^{|s_0| \cdot j} \sum_{i = 0}^{|s_0| - 1} (\overline{s_0 s_0 ... s_0}[i + |s_0| \cdot j] + 'a' - 1) \cdot p^{i} = \\
    &   = \sum_{j = 0}^{n - 1} p^{|s_0| \cdot j} \sum_{i = 0}^{|s_0| - 1} (s_0[i] + 'a' - 1) \cdot p^{i} 
        = \sum_{j = 0}^{n - 1} p^{|s_0| \cdot j} \cdot \Hash(s_0)
        = \Hash(s_0) \cdot \sum_{j = 0}^{n - 1} p^{|s_0| \cdot j} \\
    & \hash(s_0) = 0 
        \implies \Hash(s_0) \bmod 2^{64} = 0 
        \implies \hash(\overline{s_0 s_0 ... s_0}) = \Hash(\overline{s_0 s_0 ... s_0}) \bmod 2^{64} = \\
    & = \left(\Hash(s_0) \cdot \sum_{j = 0}^{n - 1} p^{|s_0| \cdot j}\right) \bmod 2^{64} = 0 \\
\end{split}
}
}

\pagebreak

\mclemma{Лемма 3}{
    Пусть существует такая строка $s_0 \in \Sigma^+$ длины $|s_0|$, что $ \hash(s_0) = 0 $

    Тогда $ \forall s \in \Sigma^* \, \forall n \in \Nset[]: \hash(\overline{s \underset{n \text{ раз}}{s_0 s_0 ... s_0}}) = \hash(s) $
\mcprf{
\begin{split}
    & \text{1. По построению хэш функции: } \\
    & \Hash(\overline{s s_0 s_0 ... s_0}) 
        = \sum_{i = 0}^{|s| + n \cdot |s_0| - 1} (\overline{s s_0 s_0 ... s_0}[i] - 'a' + 1) \cdot p^i = \\
    &   = \sum_{i = 0}^{|s| - 1} (\overline{s s_0 s_0 ... s_0}[i] + 'a' - 1) \cdot p^i 
        + \sum_{i = |s|}^{|s| + n \cdot |s_0| - 1} (\overline{s s_0 s_0 ... s_0}[i] + 'a' - 1) \cdot p^i  \\
    &   = \sum_{i = 0}^{|s| - 1} (s[i] + 'a' - 1) \cdot p^i 
        + p^{|s|} \cdot \sum_{i = |s|}^{|s| + n \cdot |s_0| - 1} (\overline{s s_0 s_0 ... s_0}[i] + 'a' - 1) \cdot p^{i - |s|} = \\
    &   = \Hash(s) + p^{|s|} \cdot \sum_{i = |s|}^{|s| + n \cdot |s_0| - 1} (\overline{s s_0 s_0 ... s_0}[i] + 'a' - 1) \cdot p^{i - |s|} = \\
    &   = \Hash(s) + p^{|s|} \cdot \sum_{i = 0}^{n \cdot |s_0| - 1} (\overline{s s_0 s_0 ... s_0}[i + |s|] + 'a' - 1) \cdot p^i = \\
    &   = \Hash(s) + p^{|s|} \cdot \sum_{i = 0}^{n \cdot |s_0| - 1} (\overline{s_0 s_0 ... s_0}[i] + 'a' - 1) \cdot p^i = \\
    &   = \Hash(s) + p^{|s|} \cdot \Hash(\overline{s_0 s_0 ... s_0}) \\
    & \text{2. по лемме 2 $\hash(\overline{s_0 s_0 ... s_0}) = 0 $, тогда:} \\
    & \hash(\overline{s s_0 s_0 ... s_0}) 
        = \Hash(\overline{s s_0 s_0 ... s_0}) \bmod 2^{64}
        = \Hash(s) \bmod 2^{64} + \left(p^{|s|} \cdot \Hash(\overline{s_0 s_0 ... s_0})\right) \bmod 2^{64} = \\
    &   = \hash(s) + \left(p^{|s|} \cdot \hash(\overline{s_0 s_0 ... s_0})\right) \bmod 2^{64} 
        = \hash(s) \\
\end{split}
}
}

\mccorollar{Следствие}{
    Если существует хотя бы одна строка $s_0 \in \Sigma^*$, такая что $ |s_0| \ge 1 \wedge \hash(s_0) = 0$, то
    $\forall s \in \Sigma^*$ можно получить хотя бы счётно бесконечно много попарно различных строк, таких
    что их хэш равен $\hash(s)$, причём алгоритм построения искомой последовательности строк следует из леммы 3:
    члены последовательности получаются конкатенацией строки $s$ с $n$ строками $s_0$.
}

\mcclarf{Уточнение об имплементации алгоритма}{
    Если длина выбранной строки $s_0$ (для которой $\hash(s_0) = 0$) равна $|s_0|$ и нужно сгенерировать $N$ искомых 
    строк с хэшом, равным данной строке $s$ длины $|s|$, то асимптотически точная граница времени работы приведённого 
    ниже алгоритма равна $ \Theta\left(|s| + N |s_0|\right)$

    В приведённом ниже алгоритме строка $s_0$ (можно генерировать разные в зависимости от шаблонного параметра) 
    имеет длину не более $\textit{kMaxZeroRemStringSize} = 20$ и генерируется на этапе компиляции (если версия языка $\ge$ C++23).
}

\pagebreak

\section*{Practice}

Алгоритм реализован в файле \textit{collisions\_gen.hpp}

Публичный интерфейс для генерации строк находится в пространстве имён \text{collisions\_gen}

Реализация находится во вложенном пространстве имён collisions\_gen::impl

Публичный интерфейс:

\begin{tabular}{rl}
    $\bullet$ & \textit{polynomial\_hash} - данная хэш функция \\
    $\bullet$ & \textit{polynomial\_hash\_safe} - данная хэш функция, использующая беззнаковую арифметику \\
    $\bullet$ & \textit{generate\_strings\_with\_same\_hash} - функция для генерации \\
              & необходимого количества строк, хэш которых равен хэшу данной строки \\
              & Кроме 2 входных аргументов имеет шаблонный параметра Seed, который \\
              & может изменить генерирующиеся строки. По умолчанию равен \textit{impl::kDefaultStartN = 1} \\
              & следующее по возрастанию значение, меняющее выходные данные $ = 13$ \\
\end{tabular}

\begin{cppcode}

namespace collisions_gen {

/// @brief Default number of strings generated by @fn generate_strings_with_same_hash
inline constexpr uint32_t kDefaultSize = 2000;

/// @brief Polynomial hash function.
///        This function causes UB and hence not marked constexpr.
/// @param s
/// @return polynomial hash of the @a `s`
long long polynomial_hash(std::string_view s) noexcept;

/// @brief Polynomial hash function.
/// @param s
/// @return polynomial hash of the @a `s`
constexpr uint64_t polynomial_hash_safe(std::string_view s) noexcept;

/// @brief Functions that generates @a `size` strings with the same
///        hash as @a `str` has. Hash functions is @fn polynomial_hash.
/// @tparam Seed optional argument that may change generated set of string.
/// @param str Initial string.
///            If empty, and Seed = @ref impl::kDefaultStartN,
///            then @a `size` strings with hash = 0 will be generated.
/// @param size Number of strings that should be generated.
/// @return vector of @a `size` pairwise different strings,
///         each one has the same hash as @a `str` has.
template <uint32_t Seed = impl::kDefaultStartN>
std::vector<std::string> generate_strings_with_same_hash(std::string_view str = "", 
uint32_t size = kDefaultSize);

}  // namespace collisions_gen

\end{cppcode}


Пространство имён impl:

(некоторые детали реализации, например, поддержка более старых версий языка, убраны в коде ниже, полный код есть в файле collisions\_gen.hpp)

\begin{tabular}{rl}
    $\bullet$ & \textit{find\_zero\_rem\_num} - функция нахождения такого n, что все цифры числа $n \cdot 2^{64}$  \\   
              & в с.с. по основанию p принадлежат $\sigma$ \\
    $\bullet$ & \textit{ZeroHashStringBuffer} - шаблонный класс, который по данному числу $n$ переводит число $n \cdot 2^{64}$ \\   
              & в с.с. по основанию p и сохраняет в виде строки \\
    $\bullet$ & \textit{zero\_hash\_string} - шаблонная константа типа std::string\_view, такая, что её хэш равен 0 \\
\end{tabular}

\begin{cppcode}

namespace impl {

using uint128_t = __uint128_t;

inline constexpr uint32_t kPrime         = 31;
inline constexpr uint32_t kDefaultStartN = 1;
/// @brief @f[ \lceil log_{kPrime}( 2^{32} \cdot 2^{64} ) \rceil @f]
inline constexpr uint32_t kMaxZeroRemStringSize = 20;

/// @brief f: { 0, ..., 2^32 - 1 } -> { 0, 1 }
/// @param n
/// @return n |-> n \in [1; 26]
constexpr bool between_1_and_26(uint32_t n) noexcept;

/// @brief Checks whether all digits of n
///        in base @ref kPrime \in [1; 26]
/// @param n
/// @return
constexpr bool check_n_in_base_p(uint128_t n) noexcept;

/// @brief Finds integer n, such that all digits
///        of n * 2^64 in base @ref kPrime \in [1; 26]
/// @tparam StartN
/// @return n
template <uint32_t StartN = kDefaultStartN>
constexpr uint32_t find_zero_rem_num() noexcept;

template <uint32_t N>
struct ZeroHashStringBuffer {
    consteval ZeroHashStringBuffer() noexcept;
    consteval std::string_view as_string_view() const noexcept;
};

template <uint32_t StartN = kDefaultStartN>
inline constexpr ZeroHashStringBuffer<find_zero_rem_num<StartN>()> zero_hash_string_buffer;

template <uint32_t StartN = kDefaultStartN>
inline constexpr std::string_view zero_hash_string 
    = zero_hash_string_buffer<StartN>.as_string_view();

}  // namespace impl

\end{cppcode}

Вызов генератора строк находится в файле \textit{main.cpp}

В качестве первого аргумента исполняемому файлу можно передать начальную строку, к которой будут дописываться строки $s_0$

В качестве второго аргумента исполняемому файлу можно передать количество строк, которое надо сгенерировать

(также можно не передавать никаких параметров, будут выбраны дефолтные: пустая строка и 2000 строк)

\begin{cppcode}

void write_to_file(const std::vector<std::string>& strs, std::string_view fname);

std::pair<const char*, uint32_t> parse_arguments(int argc, const char* const argv[]) noexcept;

int main(int argc, const char* const argv[]) {
    auto [initial_string, size] = parse_arguments(argc, argv);
    auto res = collisions_gen::generate_strings_with_same_hash(initial_string, size);
    write_to_file(res, "strings.txt");
    return 0;
}

\end{cppcode}

\pagebreak

\section*{Tested compilers \& options}

При компиляции компилятором g++ 13.2.0 из среды msys2 на Windows 10 22H2 использовались следующие флаги:

-std=c++20, -std=c++2a, -std=c++2b для версий языка C++20 и C++23, а также:

\begin{cppcode}
-D_GLIBCXX_DEBUG
-D_GLIBCXX_DEBUG_PEDANTIC
-D_FORTIFY_SOURCE=3
-fdiagnostics-color=always
-fstack-protector-all
-mshstk
-Wall
-Wextra
-Wfloat-equal
-Wlogical-op
-Wcast-qual
-Wpedantic
-Wshift-overflow=2
-Wduplicated-cond
-Wunused -Wconversion
-Wunsafe-loop-optimizations
-Wshadow
-Wnull-dereference
-Wundef 
-Wwrite-strings
-Wsign-conversion
-Warith-conversion
-Wmissing-noreturn
-Wunreachable-code
-Wcast-align
-Warray-bounds=2
\end{cppcode}

При компиляции компилятором clang++ 16.0.5 из среды msys2 на Windows 10 22H2 использовались следующие флаги:

-std=c++20, -std=c++2a, -std=c++2b для версий языка C++20 и C++23, а также:

\begin{cppcode}
-fcolor-diagnostics
-fansi-escape-codes
-fsanitize="address,undefined"
-fstack-protector-all
-D_FORTIFY_SOURCE=3
-Wp,-D_GLIBCXX_DEBUG
-mshstk
-O2
-Wall
-Wextra
-Wpedantic
-Wunused
-Wconversion
-Wshadow
-Wnull-dereference
-Wundef
-Wwrite-strings
-Wsign-conversion
-Wmissing-noreturn
-Wunreachable-code
-Wcast-align
-Warray-bounds
\end{cppcode}

При компиляции компилятором g++ 13.2.0 на Ununtu 22.10 использовались следующие флаги:

-std=c++20, -std=c++2a, -std=c++2b, -std=c++23 для версий языка C++20 и C++23, а также:

\begin{cppcode}
-D_GLIBCXX_DEBUG
-D_GLIBCXX_DEBUG_PEDANTIC
-D_FORTIFY_SOURCE=3
-fdiagnostics-color=always
-fstack-protector-all
-fsanitize="address,undefined,leak"
-mshstk
-Wall
-Wextra
-Wfloat-equal
-Wlogical-op
-Wcast-qual
-Wpedantic
-Wshift-overflow=2
-Wduplicated-cond
-Wunused -Wconversion
-Wunsafe-loop-optimizations
-Wshadow
-Wnull-dereference
-Wundef 
-Wwrite-strings
-Wsign-conversion
-Warith-conversion
-Wmissing-noreturn
-Wunreachable-code
-Wcast-align
-Warray-bounds=2
\end{cppcode}

\end{document}
