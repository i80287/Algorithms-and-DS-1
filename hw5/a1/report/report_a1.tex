\documentclass{report}
% \includeonly{pars/9_continuous_functions_and_limits, parts/10_integration_of_functions}
% \documentclass[11pt,a4paper]{article}

%%%%%%%%%%%%%%%%%%%%%%%%%%%%%%%%%
% PACKAGE IMPORTS
%%%%%%%%%%%%%%%%%%%%%%%%%%%%%%%%%

\usepackage[utf8]{inputenc} % Сурс - семинары по алгебре Медведя Никиты Юрьевича
\usepackage[T2A]{fontenc}

\usepackage[tmargin=2cm,rmargin=1in,lmargin=1in,margin=0.85in,bmargin=2cm,footskip=.2in]{geometry}
\usepackage{amsmath,amsfonts,amsthm,amssymb,mathtools}
\usepackage[varbb]{newpxmath}
\usepackage{xfrac}
\usepackage[makeroom]{cancel}
\usepackage{mathtools}
\usepackage{bookmark}
\usepackage{enumitem}
\usepackage{hyperref,theoremref}
\hypersetup{
	pdftitle={Assignment},
	colorlinks=true, linkcolor=doc!90,
	bookmarksnumbered=true,
	bookmarksopen=true
}
\usepackage[most,many,breakable]{tcolorbox}
\usepackage{xcolor}% http://ctan.org/pkg/xcolor
%\usepackage{colortbl}% http://ctan.org/pkg/colortbl
\usepackage{multirow}% http://ctan.org/pkg/multirow
\usepackage{graphicx}% http://ctan.org/pkg/graphicx
\usepackage{varwidth}
\usepackage{varwidth}
\usepackage{etoolbox}
%\usepackage{authblk}
\usepackage{nameref}
\usepackage{multicol,array}
\usepackage{tikz-cd}
\usepackage[ruled,vlined,linesnumbered]{algorithm2e}
\usepackage{comment} % enables the use of multi-line comments (\ifx \fi) 
\usepackage{import}
\usepackage{xifthen}
\usepackage{pdfpages}
\usepackage{transparent}

\usepackage[english]{babel}
% ,russian
\usepackage{amsfonts,amssymb}
\usepackage{relsize}

\usepackage{systeme}

\usepackage{indentfirst} % Красная строка
\usepackage{fancyhdr}
\usepackage{wrapfig}
\usepackage{textcomp}

% \usepackage[unicode]{hyperref}

\newcommand\mycommfont[1]{\footnotesize\ttfamily\textcolor{blue}{#1}}
\SetCommentSty{mycommfont}
\newcommand{\incfig}[1]{%
    \def\svgwidth{\columnwidth}
    \import{./figures/}{#1.pdf_tex}
}

\usepackage{tikzsymbols}
\renewcommand\qedsymbol{$\Laughey$}


%\usepackage{import}
%\usepackage{xifthen}
%\usepackage{pdfpages}
%\usepackage{transparent}


%%%%%%%%%%%%%%%%%%%%%%%%%%%%%%
% SELF MADE COLORS
%%%%%%%%%%%%%%%%%%%%%%%%%%%%%%



\definecolor{myg}{RGB}{56, 140, 70}
\definecolor{myb}{RGB}{45, 111, 177}
\definecolor{myr}{RGB}{199, 68, 64}
\definecolor{mytheorembg}{HTML}{F2F2F9}
\definecolor{mytheoremfr}{HTML}{00007B}
\definecolor{mylenmabg}{HTML}{FFFAF8}
\definecolor{mylenmafr}{HTML}{983b0f}
\definecolor{mypropbg}{HTML}{f2fbfc}
\definecolor{mypropfr}{HTML}{191971}
\definecolor{myexamplebg}{HTML}{F2FBF8}
\definecolor{myexamplefr}{HTML}{88D6D1}
\definecolor{myexampleti}{HTML}{2A7F7F}
\definecolor{mydefinitbg}{HTML}{E5E5FF}
\definecolor{mydefinitfr}{HTML}{3F3FA3}
\definecolor{notesgreen}{RGB}{0,162,0}
\definecolor{myp}{RGB}{197, 92, 212}
\definecolor{mygr}{HTML}{2C3338}
\definecolor{myred}{RGB}{127,0,0}
\definecolor{myyellow}{RGB}{169,121,69}
\definecolor{myexercisebg}{HTML}{F2FBF8}
\definecolor{myexercisefg}{HTML}{88D6D1}

%%%%%%%%%%%%%%%%%%%%%%%%%%%%%%%%%%%%%%%%%%%
% TABLE OF CONTENTS
%%%%%%%%%%%%%%%%%%%%%%%%%%%%%%%%%%%%%%%%%%%

\usepackage{tikz}
\definecolor{doc}{RGB}{0,60,110}
\usepackage{titletoc}
\contentsmargin{0cm}
\titlecontents{chapter}[3.7pc]
{\addvspace{30pt}%
	\begin{tikzpicture}[remember picture, overlay]%
		\draw[fill=doc!60,draw=doc!60] (-7,-.1) rectangle (-0.9,.5);%
		\pgftext[left,x=-3.5cm,y=0.2cm]{\color{white}\Large\sc\bfseries Chapter\ \thecontentslabel};%
	\end{tikzpicture}\color{doc!60}\large\sc\bfseries}%
{}
{}
{\;\titlerule\;\large\sc\bfseries Page \thecontentspage
	\begin{tikzpicture}[remember picture, overlay]
		\draw[fill=doc!60,draw=doc!60] (2pt,0) rectangle (4,0.1pt);
	\end{tikzpicture}}%
\titlecontents{section}[3.7pc]
{\addvspace{2pt}}
{\contentslabel[\thecontentslabel]{2pc}}
{}
{\hfill\small \thecontentspage}
[]
\titlecontents*{subsection}[3.7pc]
{\addvspace{-1pt}\small}
{}
{}
{\ --- \small\thecontentspage}
[ \textbullet\ ][]

\makeatletter
\renewcommand{\tableofcontents}{%
	\chapter*{%
	  \vspace*{-20\p@}%
	  \begin{tikzpicture}[remember picture, overlay]%
		  \pgftext[right,x=15cm,y=0.2cm]{\color{doc!60}\Huge\sc\bfseries \contentsname};%
		  \draw[fill=doc!60,draw=doc!60] (13,-.75) rectangle (20,1);%
		  \clip (13,-.75) rectangle (20,1);
		  \pgftext[right,x=15cm,y=0.2cm]{\color{white}\Huge\sc\bfseries \contentsname};%
	  \end{tikzpicture}}%
	\@starttoc{toc}}
\makeatother


\input{macros}

\begin{document}

\begin{cppcode}
INSERT(key):
    ind = hash(key) mod M
    while table[ind] != NULL
        if table[ind] == key
            return
        ind = (ind + 1) mod M
    table[ind] = key
\end{cppcode}
\begin{cppcode}
DELETE(key):
    ind = hash(key) mod M
    while table[ind] != NULL
        if table[ind] == key
            table[ind] = ERASED
            return
        ind = (ind + 1) mod M
\end{cppcode}
\begin{cppcode}
SEARCH(key):
    ind = hash(key) mod M
    while table[ind] != NULL
        if table[ind] == key
            return TRUE 
        ind = (ind + 1) mod M
    return FALSE
\end{cppcode}

БОО предположим, что функция hash для целочисленных типов реализована так же, как
call operator в функторе std::hash стандартной библиотеки языка C++, т.е. для целочисленных типов
их хэш равен их значению, проинтепретируемому как std::size\_t.

\section*{Первый пример}

Первым примером последовательности, для которой операции будут работать \textbf{\textit{долго}},
может быть такая последовательность:

INSERT(0)

DELETE(0)

INSERT(1)

DELETE(1)

INSERT(2)

DELETE(2)

\dots

INSERT(n)

DELETE(n)

для некоторого $ n < M $

Т.к. при удалении ключей они помечаются \textcolor{orange}{ERASED}, 
и хэши по модулю M вставленных ключей 0, ..., n - 1 - это целые числа 
от 0 до n - 1, то во время выполнения INSERT(n) и DELETE(n) массив table 
будет заполнен значениям \textcolor{orange}{ERASED} во всех ячейках с 
индексами 0, ..., n - 1, поэтому вставка и последующее удаление ключа 
n произойдёт за линейное относительно размера таблицы время.

Возможной доработкой для данного случая может быть изменение функции 
INSERT - возможность вставки кдюча в ячейку, которая равна не только 
\textcolor{orange}{NULL}, но и \textcolor{orange}{ERASED}

\begin{cppcode}
INSERT(key):
    ind = hash(key) mod M
    while table[ind] != NULL and table[ind] != ERASED
        if table[ind] == key
            return
        ind = (ind + 1) mod M
    table[ind] = key
\end{cppcode}

Эта небольшая доработка не испортит корректность структуры данных, т.к. 
до доработки роль элементов, сохраняющих целостность кластера при удалении, 
играли только \textcolor{orange}{ERASED}, а после неё на их место стали
вставляться действительные ключи, которые также подходят для сохранения
кластера, т.к. непосредственно являются его элементами.

\section*{Второй пример}

Вторым примером последовательности, операции над которой приведут к \textbf{\textit{долгой}} 
работе (за линейное время относительно размера таблицы), может быть:

INSERT(0)

INSERT(M)

INSERT(2 $\cdot$ M)

INSERT(3 $\cdot$ M)

INSERT(4 $\cdot$ M)

\dots

INSERT(n $\cdot$ M)

SEARCH(n $\cdot$ M)

DELETE(n $\cdot$ M)

для некоторого $ n < M $

На момент вызова операции INSERT(n$\cdot$ M), в хэщ-таблице будет $n$ элементов,
при этом, т.к. у всех вставленных ключей хэш по модулю M равен 0, то они при вставке 
последовательно вставлялись в массив table по индексам 0, 1, \dots, n - 1, поэтому 
вставка ключа n $\cdot$ M займёт линейное время относительно размера таблицы 
(мы переберём все индексы от 0 до n - 1, пока не найдём первое свободное место по индексу n)

Аналогично, для операций SEARCH(n $\cdot$ M) и DELETE(n $\cdot$ M) время их выполнения 
будет пропорционально размеру хэш-таблицы, т.е. поиск и удаление выполнятся за линейное 
время относительно количества элементов в таблице.

Для решения данной проблемы можно использовать квадратичное пробирование 
и параллельный поиск свободных значений при помощи векторных инструкций
как, например, в:

\url{https://doc.rust-lang.org/std/collections/struct.HashMap.html}

\includegraphics[scale=0.60]{img1.png}

\includegraphics[scale=0.60]{img2.png}

\end{document}
