\documentclass[11pt,a4paper]{article}

%%%%%%%%%%%%%%%%%%%%%%%%%%%%%%%%%
% PACKAGE IMPORTS
%%%%%%%%%%%%%%%%%%%%%%%%%%%%%%%%%

\usepackage[utf8]{inputenc} % Сурс - семинары по алгебре Медведя Никиты Юрьевича
\usepackage[T2A]{fontenc}

\usepackage[tmargin=2cm,rmargin=1in,lmargin=1in,margin=0.85in,bmargin=2cm,footskip=.2in]{geometry}
\usepackage{amsmath,amsfonts,amsthm,amssymb,mathtools}
\usepackage[varbb]{newpxmath}
\usepackage{xfrac}
\usepackage[makeroom]{cancel}
\usepackage{mathtools}
\usepackage{bookmark}
\usepackage{enumitem}
\usepackage{hyperref,theoremref}
\hypersetup{
	pdftitle={Assignment},
	colorlinks=true, linkcolor=doc!90,
	bookmarksnumbered=true,
	bookmarksopen=true
}
\usepackage[most,many,breakable]{tcolorbox}
\usepackage{xcolor}% http://ctan.org/pkg/xcolor
%\usepackage{colortbl}% http://ctan.org/pkg/colortbl
\usepackage{multirow}% http://ctan.org/pkg/multirow
\usepackage{graphicx}% http://ctan.org/pkg/graphicx
\usepackage{varwidth}
\usepackage{varwidth}
\usepackage{etoolbox}
%\usepackage{authblk}
\usepackage{nameref}
\usepackage{multicol,array}
\usepackage{tikz-cd}
\usepackage[ruled,vlined,linesnumbered]{algorithm2e}
\usepackage{comment} % enables the use of multi-line comments (\ifx \fi) 
\usepackage{import}
\usepackage{xifthen}
\usepackage{pdfpages}
\usepackage{transparent}

\usepackage[english]{babel}
% ,russian
\usepackage{amsfonts,amssymb}
\usepackage{relsize}

\usepackage{systeme}

\usepackage{indentfirst} % Красная строка
\usepackage{fancyhdr}
\usepackage{wrapfig}
\usepackage{textcomp}

% \usepackage[unicode]{hyperref}

\newcommand\mycommfont[1]{\footnotesize\ttfamily\textcolor{blue}{#1}}
\SetCommentSty{mycommfont}
\newcommand{\incfig}[1]{%
    \def\svgwidth{\columnwidth}
    \import{./figures/}{#1.pdf_tex}
}

\usepackage{tikzsymbols}
\renewcommand\qedsymbol{$\Laughey$}


%\usepackage{import}
%\usepackage{xifthen}
%\usepackage{pdfpages}
%\usepackage{transparent}


%%%%%%%%%%%%%%%%%%%%%%%%%%%%%%
% SELF MADE COLORS
%%%%%%%%%%%%%%%%%%%%%%%%%%%%%%



\definecolor{myg}{RGB}{56, 140, 70}
\definecolor{myb}{RGB}{45, 111, 177}
\definecolor{myr}{RGB}{199, 68, 64}
\definecolor{mytheorembg}{HTML}{F2F2F9}
\definecolor{mytheoremfr}{HTML}{00007B}
\definecolor{mylenmabg}{HTML}{FFFAF8}
\definecolor{mylenmafr}{HTML}{983b0f}
\definecolor{mypropbg}{HTML}{f2fbfc}
\definecolor{mypropfr}{HTML}{191971}
\definecolor{myexamplebg}{HTML}{F2FBF8}
\definecolor{myexamplefr}{HTML}{88D6D1}
\definecolor{myexampleti}{HTML}{2A7F7F}
\definecolor{mydefinitbg}{HTML}{E5E5FF}
\definecolor{mydefinitfr}{HTML}{3F3FA3}
\definecolor{notesgreen}{RGB}{0,162,0}
\definecolor{myp}{RGB}{197, 92, 212}
\definecolor{mygr}{HTML}{2C3338}
\definecolor{myred}{RGB}{127,0,0}
\definecolor{myyellow}{RGB}{169,121,69}
\definecolor{myexercisebg}{HTML}{F2FBF8}
\definecolor{myexercisefg}{HTML}{88D6D1}

%%%%%%%%%%%%%%%%%%%%%%%%%%%%%%%%%%%%%%%%%%%
% TABLE OF CONTENTS
%%%%%%%%%%%%%%%%%%%%%%%%%%%%%%%%%%%%%%%%%%%

\usepackage{tikz}
\definecolor{doc}{RGB}{0,60,110}
\usepackage{titletoc}
\contentsmargin{0cm}
\titlecontents{chapter}[3.7pc]
{\addvspace{30pt}%
	\begin{tikzpicture}[remember picture, overlay]%
		\draw[fill=doc!60,draw=doc!60] (-7,-.1) rectangle (-0.9,.5);%
		\pgftext[left,x=-3.5cm,y=0.2cm]{\color{white}\Large\sc\bfseries Chapter\ \thecontentslabel};%
	\end{tikzpicture}\color{doc!60}\large\sc\bfseries}%
{}
{}
{\;\titlerule\;\large\sc\bfseries Page \thecontentspage
	\begin{tikzpicture}[remember picture, overlay]
		\draw[fill=doc!60,draw=doc!60] (2pt,0) rectangle (4,0.1pt);
	\end{tikzpicture}}%
\titlecontents{section}[3.7pc]
{\addvspace{2pt}}
{\contentslabel[\thecontentslabel]{2pc}}
{}
{\hfill\small \thecontentspage}
[]
\titlecontents*{subsection}[3.7pc]
{\addvspace{-1pt}\small}
{}
{}
{\ --- \small\thecontentspage}
[ \textbullet\ ][]

\makeatletter
\renewcommand{\tableofcontents}{%
	\chapter*{%
	  \vspace*{-20\p@}%
	  \begin{tikzpicture}[remember picture, overlay]%
		  \pgftext[right,x=15cm,y=0.2cm]{\color{doc!60}\Huge\sc\bfseries \contentsname};%
		  \draw[fill=doc!60,draw=doc!60] (13,-.75) rectangle (20,1);%
		  \clip (13,-.75) rectangle (20,1);
		  \pgftext[right,x=15cm,y=0.2cm]{\color{white}\Huge\sc\bfseries \contentsname};%
	  \end{tikzpicture}}%
	\@starttoc{toc}}
\makeatother



% \addtolength{\textwidth}{130pt}
% \addtolength{\hoffset}{-2cm}
% \addtolength{\voffset}{-2cm}
% \addtolength{\textheight}{90pt}

% \tolerance=3000
% \def\baselinestretch{1.1}
% \flushbottom

% \parindent=1cm

\begin{document}

\section*{Задание A1}

1. Для вычисления приблизительного значения числа $ \pi $ была разработана функция \\
\textit{CalcApproxPi}, которая вызывает вспомогательные функции \textit{GeneratePointsInSquare} и \\
\textit{CountPointsInCircle}. Запись значений в файлы происходит в функции \textit{WriteResultsToFiles}.

    Для построения графиков в функции \textit{PlotGraphs} используется приложение \textit{gnuplot}, \\
команды к которому передаются через \textit{pipe}, а данные - через файлы с записанными ранее \\
результатами измерений.

    Для воспроизводимости измерений генератор псевдослучайных чисел (\textit{std::mt19937}) \\
инициализируется константным значением \textit{kRndSeed}.

    Данные сохраняются в файлы pi\_values\_tests и percntg\_diff\_tests.data

\begin{figure}[hp]
    \centering
    \includegraphics[scale=0.65]{a1_img1.png}
\end{figure}

(компилируемость кода проверялась при помощи компилятора \textit{g++} версии 13.2.0 с флагами компиляции:
-std=c++2b -Wall -Wextra -Wpedantic -Werror -Wunused -pedantic-errors -Wconversion -Wshadow -Wnull-dereference -Warith-conversion -Wcast-align=strict -Warray-bounds=2)

(использованная версия \textit{gnuplot}: gnuplot 5.4 patchlevel 8)

\pagebreak

2. По результатам проведённых экспериментов было построено 2 графика:

\begin{tabular}{rl}
    & $\bullet$ зависимость приблизительного значения числа $\pi(N)$ от $N$ (график 1) \\
    & $\bullet$ зависимость относительного отклонения (в \%) от $N$ (график 2) \\ 
    & \,\,\,\, (вычисляется по формуле $ \left| \frac{\pi(N) - \pi}{\pi} \right| * 100 = \left| 1 - \frac{\pi(N)}{\pi} \right| * 100 $) \\
\end{tabular}

График 1:

На данном графике синяя горизонтальная прямая - значение числа $ \pi $ (с точностью, которую позволяет получить стандарт IEEE-754)

\hspace*{-2cm} \includegraphics[scale=0.5]{pi_values_graph.PNG}

График 2:

\hspace*{-2cm} \includegraphics[scale=0.5]{percentage_diff_graph.PNG}

3. Как видно из графиков, точность, с которой вычисляется значение числа $\pi$, в среднем увеличивается 
с увеличением количества точек, однако при бОльших значениях $N$ значение $\pi(N)$ медленнее приближается к числу $\pi$

\end{document}
